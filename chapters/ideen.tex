\section{Ideen}
\subsection{Einleitung}
Benlian SaaS 2010: Chancen und Risiken für diesen Anwendungsfall aus Anwendersicht prüfen. In Softwareindustrie werden die einzelnen Chancen und Risiken genauer ausgeführt. ~S. 236
Auch Chancen und Risiken aus Anbietersicht. Softwareindustrie: S: 240 => Verweis auf Benlian 2010, S. 233 \\
Wind: Eval. und Auswahl von Enterprise Cloud Services: Konkretisierung von Zieldimensionen (Flexibilität, Kosten, Leistungsumfang  \& Leistungsfähigkeit, Service \& Cloud Management, IT-Sicherheit \&
Compliance, Ausfallsicherheit \& Vertrauenswürdigkeit). Ab Seite 103. Anforderungsrahmen für die Zieldimensionen und den einzelnen XaaS-Arten. Ab S. 122. Viele Definitionen für Cloud.

Benlian Opportunities and risks of saas 2011: Sicherheit ein Hauptfaktor bei Entscheidung für oder gegen SaaS. Taxonomy of it security risks als Checkliste zur Identifikation von Risiken in bestimmtren 
Szenarien. \\

Softwareindustrie: Simple Definition für Cloud aus Standard. \\

Ackermann, Tobias: IT Security Risk Management. Kapitel 5 enthält Empfehlungen für Risk Identification, - Quantification, - Treatment, - Review and Evaluation, - Cloud Computing Providers. S. 22-23 enthält
Beschreibung der Risiken im Cloud Kontext.


\subsection{Gamechanger Cloud}
"`What is it about cloud that makes it a game-changer? It
is reported that the business at large find cloud’s affordable,
flexible, on-demand, elastic delivery method, to be extremely
beneficial"' 
-- \pcite{}{}{cloud_based_next_generation_service_and_key_challenges}

Durch Nutzung der Cloud-Plattform stehen drei infrastrukturelle Elemente zur 
Verfügung: das Soziale, Möglichkeiten der Analyse und das Mobile. Durch das 
Soziale wird die angebotene Dienstleistung inklusiver; Nutzer lassen sich 
besser in die Weiterentwicklung des Dienstes einbeziehen. \\
Da alle Kunden auf der Cloud arbeiten, lässt sich eine Vielzahl von 
Informationen über sie sammeln mit der zukünftige Enstcheidungen fundierter 
getroffen werden können. \\
Die Möglichkeit Dienste mobil in Anspruch zu nehmen, sorgt für eine höhere 
Marktdurchdringung. 
\pcite{}{}{cloud_based_next_generation_service_and_key_challenges}

\subsubsection{Lifecycle}
Aus \citeflow{cloud_based_next_generation_service_and_key_challenges}
\begin{description}
	\item[Design] RE und Design des Dienstes. 
	\begin{description}
	\item[Service Composition] Es gibt einen Trend dazu, Dienste mehrerer 
Cloud-Dienstleister zusammenzustellen. Wie auch im On-Premise-Bereich ist auf 
eine lose Kopplung der Komponenten zu achten, um sie bei Bedarf gegen 
Alternativen tauschen zu können. Außerdem ist es wichtig, das Thema 
Fehlertoleranz zu Bedenken. Damit Ausfälle einzelner Komponenten sich möglichst 
wenig auf den Dienst auswirken oder sich Ausfallwahrscheinlichkeiten 
aufaddieren.
	\item[Interoperabilität] Unter anderem Amazon, Google, Microsoft, 
VMWare und Salesforce bieten Cloud-Plattformen zur Anwendungsentwicklung an. 
Bei Nutzung einer solchen Plattform kann nachhaltige Interoperabilität, gerade 
bei Nutzung einer privaten oder hybriden Cloud ein Faktor sein.
	\item[Information Centric Design] Bei der Nutzung durch Kunden 
anfallende Informationen sollten ständig ausgewertet werden und möglichst in 
Echtzeit in zukünftige Entscheidungen einfließen. 
\end{description}
	\item[Engineering] Entwicklung, Testen und Operationalisierung \\
	\item[Deployment] 
	\item[Usage Measurement]
	\item[Support und Wartung] Informationen, die im Support gesammelt 
werden, werden zur Weiterentwicklung genutzt.
	\item[Experience] Um Kunden zu halten, muss die User Experience 
verbessert werden
\end{description}

\subsubsection{Service Migration -- Herausforderungen}
Aus \citeflow{cloud_based_next_generation_service_and_key_challenges}
\begin{itemize}
	\item Was für Auswirkungen hat das neue pay-per-use-Modell?
	\item 
\end{itemize}

\subsection{Charakteristika Enterprise Applications}
Aus \citeflow{cloudward_bound_planning_for_beneficial_migration}:\\
Bestehen in der Regel aus mehreren Layern (MVC -- Datenbanken, Frontend, 
Logik), sind in der Regel aber viel komplexer als dieses dreischichtige Modell, 
da jede Schicht aus mehreren, miteinander interagierenden Komponenten aufgebaut 
sein kann. \\
Unternehmensanwendungen könnten von zwei verschiedenen Nutzergruppen nutzbar 
sein: Unternehmensinterne und -externe Personengruppen.

\subsection{Assessment und Guidelines}
Aus \citeflow{challenges_and_assessment_in_migrating}: \\
\subsubsection{Assessment}
\begin{description}
	\item[Gründe für die Migration] Es ist wichtig die Gründe für die 
Migration sowie Anforderungen und Rahmenbedingungen zu kennen, um ihnen zu 
genügen. 
	\item[Analyse der Anwendungsumgebung] Alle Programme, Skripte und 
Interfaces listen, die auf die Anwendung zugreifen.
	\item[Analyse der neuen Umgebung] Welche Ressourcen werden benötigt?
	\item[Design und Analyse der Architektur] Analyse der Architektur der 
bestehenden Anwendung mit allen Bibliotheken, Programmen und Plattformen 
	\item[Migrationstools] Kleine Proof-of-Concept-Projekte um 
Migrationstoosl nach Effizienz, Genauigkeit und Optionen testen.
\end{description}
\subsubsection{Guidelines} 
\begin{description}
	\item[Interoperabilität und Zusammenstellen] Nutzen von Schnittstellen 
wie der RESTful API.
	\item[Vermeiden des Lock-in Effektes]
	\item[Modellierung des Dienstes und Cloud Deployment Artefakte] Cloud 
Modellierungsnotationen sind nützlich. 
	\item[Legacy Software verpacken] Es ist unter Umständen einfacher, 
praktikabler, sicherer und flexibler Legacy Anwendungen virtualisiert in die 
CLoud zu migrieren und über eine Schnittstelle verfügbar zu machen.
	\item[Ausbalancieren von Kosten und Reifegrad] Die Risiken junger, 
eventuell unreifer Software sollten in den Aspekt Kosten einbezogen werden.
\end{description}



\subsection{Inhaltsbeschreibungen}
\begin{description}
	
\item[\citeflow{cloud_based_next_generation_service_and_key_challenges}] Viele 
zu bedenkende Aspekte, nicht nur für die Migration, sondern auch für den 
Cloudbetrieb. Eine Zusammenfassung findet sich auf Seite 9 des PDFs.
\item[\citeflow{cloudward_bound_planning_for_beneficial_migration}] Hybride 
Cloud. Darstellung Charakteristike Enterprise Applications. Sonst extrem 
mathematisch.
\end{description}


