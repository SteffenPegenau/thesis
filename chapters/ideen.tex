\begin{comment}
\section{Ideen}
\subsection{Einleitung}
Benlian SaaS 2010: Chancen und Risiken für diesen Anwendungsfall aus Anwendersicht prüfen. In Softwareindustrie werden die einzelnen Chancen und Risiken genauer ausgeführt. ~S. 236
Auch Chancen und Risiken aus Anbietersicht. Softwareindustrie: S: 240 => Verweis auf Benlian 2010, S. 233 \\
Wind: Eval. und Auswahl von Enterprise Cloud Services: Konkretisierung von Zieldimensionen (Flexibilität, Kosten, Leistungsumfang  \& Leistungsfähigkeit, Service \& Cloud Management, IT-Sicherheit \&
Compliance, Ausfallsicherheit \& Vertrauenswürdigkeit). Ab Seite 103. Anforderungsrahmen für die Zieldimensionen und den einzelnen XaaS-Arten. Ab S. 122. Viele Definitionen für Cloud.

Benlian Opportunities and risks of saas 2011: Sicherheit ein Hauptfaktor bei Entscheidung für oder gegen SaaS. Taxonomy of it security risks als Checkliste zur Identifikation von Risiken in bestimmtren
Szenarien. \\

Ackermann, Tobias: IT Security Risk Management. Kapitel 5 enthält Empfehlungen für Risk Identification, - Quantification, - Treatment, - Review and Evaluation, - Cloud Computing Providers. S. 22-23 enthält
Beschreibung der Risiken im Cloud Kontext. \\

Key players in the cloud computing industry 
\pcite{}{4}{cloud-computing_the_business_perspective}\\



\subsection{Gamechanger Cloud}
"`What is it about cloud that makes it a game-changer? It
is reported that the business at large find cloud’s affordable,
flexible, on-demand, elastic delivery method, to be extremely
beneficial"'
-- \pcite{}{}{cloud_based_next_generation_service_and_key_challenges}


\subsubsection{Towards Modelling a Cloud Applications Life Cycle}
Aus \citeflow{towards_modelling_a_cloud_applications_life_cycle}
\begin{itemize}
	\item Bisherige Life Cycle Modelle fokussieren die Sicht der IT. Es fehlt
	ihnen an einer angemessenen Betrachtung wirtschaftlicher Aspekte.
	\item Literatur zu Cloud und Marketing konzentriert sich auf technische
	und finanzielle Faktoren im Deployment und vernachlässigt die Supply Chain
	des Unternehmens und die Auswirkungen auf das Unternehmen selbst
	\item Die Cloud ermöglicht es einem Unternehmen unabhängig von seiner Größe,
	spezialisierte, hochwertige IT-Dienstleistungen in Anspruch zu nehmen.
	\item Die Cloud ist eine ``disruptive innovation'', ein ``new Market'', auf dem IT als Dienstleistung eingekauft wird. Dieser Markt erfordert einen grundlegenden, kulturellen Wandel, mit dem die IT gesehen wird.
	\item Opportunities and Impact
	\begin{itemize}
		\item Verpasste Gelegenheiten durch Über- oder Unterprovisionierung vermeiden
		\item An variable Nachfrage angepasste Dienstleistung
		\item Das Verfolgen von wachsenden oder völlig neuen Märkten
		\item Tätigkeiten schneller und günstiger durchführen
		\item Entkoppeln des Kerngeschäfts von unterstützenden Dienstleistungen
		\item Höhere Umsätze durch amortisierende Skaleneffekte
		\item Höhere Umsätze durch höhere Reichweite
	\end{itemize}
	\item Unternehmen, die in die Cloud gehen, müssen sich den Auswirkungen auf Strategie, Business Model und Unternehmensstruktur bewusst sein.
\end{itemize}

\subsection{Inhaltsbeschreibungen}
\begin{description}

\item[\citeflow{cloud_based_next_generation_service_and_key_challenges}] Viele
zu bedenkende Aspekte, nicht nur für die Migration, sondern auch für den
Cloudbetrieb. Eine Zusammenfassung findet sich auf Seite 9 des PDFs.
\item[\citeflow{cloudward_bound_planning_for_beneficial_migration}] Hybride
Cloud. Darstellung Charakteristike Enterprise Applications. Sonst extrem
mathematisch.
\end{description}
\end{comment}


