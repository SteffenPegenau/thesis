\section{Entwicklung eines konzeptuellen Rahmens}
In diesem Kapitel soll das Fünf-Phasen-Modell auf die ANforderungen eines 
Independent Software Vendors angepasst werden.

\subsection{Phase 1: Machbarkeitsstudie}
\subsubsection{Technische Machbarkeit}
\subsubsection{Wirtschaftliche Machbarkeit}

\subsection{Phase 2: Anforderungsanalyse und -Planung}
\subsubsection{Anforderungsermittlung}
\subsubsection{Return on Investment (ROI)}
\subsubsection{Total Cost of Ownership (TCO)}

\subsection{Phase 3: Migration}
\subsection{Phase 4: Tests und Auslieferung}
\subsection{Phase 5: Überwachung und Wartung}
\subsection{Gesamtbetrachtung}
\subsubsection{Agilität}



\begin{comment}
\label{cha:entwicklung}
Dieses Kapitel dient der Entwicklung eines konzeptuellen Rahmens auf Basis theoretischer Grundlagen, vorausgesetzt sie verfolgen einen positivistischen Ansatz. Hierfür leiten Sie Hypothesen aus verschiedenen sinnvoll kombinierten Quellen her. Hierdurch generieren Sie aus bestehendem Wissen neues Wissen, was eine Eigenleistung und somit ein wichtiger Bestandteil Ihrer Arbeit darstellt.

Sollte Ihre Arbeit nicht positivistisch ausgelegt sein, stellt dieser Abschnitt kein Pflichtkapitel der Arbeit dar. Alternativ beschreiben Sie Anforderungen für ein mögliches Konzept oder verzichten vollständig auf dieses Kapitel.

\textbf{Setzen Sie sich frühzeitig mit Ihrem Betreuer in Verbindung, um Ihre Gliederung abzustimmen und mögliche Missverständnisse zu beseitigen.}

Im Folgenden werden einige allgemeine Hinweise zu den Themen richtiges Zitieren und Literaturrecherche gegeben.


\subsection{Quellen und richtiges Zitieren}
Quellen können in Fußnote oder direkt im Text platziert werden. Alles was nicht Ihr eigenes Gedankengut darstellt, muss mit einer entsprechenden Quelle belegt werden. Hierbei können wörtliche und indirekte Zitate verwendet werden. Wörtliche Zitate sind immer mit der Seitennummer der Quelle anzugeben.

Beispiel für ein direktes Zitat:

%\textit{\glqq The case study is a research strategy which focuses in understanding the dynamics present within single settings\grqq} \pcite{}{543}{eisenhardt1989}.

Beispiel für ein indirektes Zitat:

%Eine explorative Fallstudie dient der Gewinnung von neuen Erkenntnissen und der Bildung von neuen Hypothesen über bestimmte Sachverhalte. Durch den Beitrag zum Theorieaufbau ist der Erkenntnisgewinn höher als bei einer reinen deskriptiven Fallstudie. In explorativen Fallstudien werden Phänomene in noch wenig erforschten Gebieten identifiziert und aus erkannten Zusammenhängen neue Hypothesen gebildet \citepara{eisenhardt1989}.

Alternativ kann die Quelle auch im  laufenden Text angegeben werden:

%Nach \citeflow{eisenhardt1989} wird die Wichtigkeit der Fallauswahl oft unterschätzt. Die Fälle können zwar zufällig ausgewählt werden, dies ist aber weder notwendig noch wünschenswert.

Quellenangaben bestehen aus Autor, Jahr und ggf. Seitenangabe. Bei zwei Autoren sind beide Autoren zu nennen, bei mehreren Autoren nur der erste Autor mit dem Zusatz „et al.“.\newpage


\subsection{Zitieren mit Endnoten}
Im Rahmen der Erstellung von Arbeiten am Fachgebiet ISE ist das Literaturverwaltungsprogramm EndNote zu verwenden. Dieses steht auf der \href{http://www.ulb.tu-darmstadt.de/service/literaturverwaltung_start/endnote_ulb/endnote.de.jsp
}{ULB-Seite} zum Download verfügbar. 


\subsubsection{Lateinischer Text mit Zitaten für Erstellung des Literaturverzeichnisses}
\label{cha:source:latintext}
%Lorem ipsum dolor sit amet, consectetur adipiscing elit. Sed vitae lacus eu 
%augue semper lobortis vitae aliquet leo. Fusce eleifend sodales commodo 
%\citepara{eisenhardt1989}. Mauris arcu metus, bibendum sagittis condimentum 
%eget, placerat a enim \citepara{baechle2010}. Quisque sit amet sagittis lectus. 
%Curabitur sit amet libero eu felis elementum mollis. Nullam odio diam, mollis 
%vitae viverra ut, laoreet ut odio. Praesent facilisis suscipit consequat. Morbi 
%feugiat rutrum erat, eu sagittis nibh rhoncus nec \citepara{melao2000}.

%In euismod, arcu ut semper adipiscing, nibh odio ullamcorper arcu, ut scelerisque massa magna nec quam \citepara{benlian2013}. Curabitur bibendum nibh eget augue pellentesque iaculis \citepara{sheffi2005}. Praesent iaculis auctor gravida. Quisque congue, magna ut bibendum semper, enim tortor ultrices lorem, ac feugiat tortor lectus nec nunc \citepara{carnap1974}. Pellentesque habitant morbi tristique senectus et netus et malesuada fames ac turpis egestas. Lorem ipsum dolor sit amet, consectetur adipiscing elit. Fusce dignissim, augue a sodales tristique, neque dui mollis arcu, id interdum augue justo sed lacus \citepara{welchering2013}. Vestibulum ante ipsum primis in faucibus orci luctus et ultrices posuere cubilia Curae; Mauris euismod bibendum nulla, sed accumsan urna tempor sed. Etiam eget diam eros, sed aliquet dolor \citepara{broadbent1996}. Phasellus vitae quam in orci convallis pharetra. Donec sit amet imperdiet nisi \citepara{kayser2013}. Sed vel interdum orci. Praesent vulputate, dolor id varius egestas, enim libero cursus neque, a cursus sapien nulla ut augue. Nullam vitae tortor nisl, vitae cursus enim. Suspendisse eget metus ipsum, sit amet varius sem \citepara{shazly2013}.

\subsection{Literaturrecherche}
Anbei eine kurze Auflistung von möglichen Kanälen zur Literaturrecherche.

\textbf{Zu Verwaltung Ihrer Literatur benutzen Sie bitte das Programm EndNote, dieses wird kostenfrei von der TU zu Verfügung gestellt.}

\url{http://www.ulb.tu-darmstadt.de/angebot/service/literaturverwaltung/endnote.de.jsp}

\subsubsection{Angebot der ULB}
\begin{itemize}
\item Universitätsbibliotheken (\url{http://www.ulb.tu-darmstadt.de/})
\item Rechercheangebot der ULB (\url{http://www.ulb.tu-darmstadt.de/recherche/})
\end{itemize}

\subsubsection{Online-Datenbanken und -Bibliotheken}
\begin{itemize}
\item Elektronische Zeitschriftenbibliothek (EZB) \\
(\url{http://rzblx1.uni-regensburg.de/ezeit/fl.phtml?bibid=TUDA})
\item AIS Electronic Library (AISeL)\\
(\url{http://aisel.aisnet.org/})
\item Zeitschriftendatenbank (ZDB)\\
(\url{http://dispatch.opac.ddb.de/DB=1.1/srt=YOP/})
\item Datenbank-Infosystem (DBIS): Literatur- und Fakten-Datenbank\\
(\url{http://rzblx10.uni-regensburg.de/dbinfo/fachliste.php?bib_id=tud})
\item IEEE Xplore \\
(\url{http://ieeexplore.ieee.org/Xplore/dynhome.jsp?tag=1})
\item EBSCO: internationale wirtschafts-wiss. Zeitschriften\\ (\url{http://search.ebscohost.com})
\item Springer-Online: Bücher/Beiträge des Springer Verlags\\
(\url{http://www.springerlink.com})
\item WiSo Net: deutschsprachige Literatur zu Wirtschafts- und Sozialwissenschaften\\
(\url{www.wiso-net.de})

\end{itemize}

\subsubsection{Sonstiges}
\begin{itemize}
\item \textbf{Google Scholar:} Suchdienst für wissenschaftliche Recherchen (http://scholar.google.de)
\item \textbf{Verlagswebseiten} Recherche und den Zugriff auf Zeitschriften- und Zeitungsartikel und E-Books
\item \textbf{Webseiten von Unternehmen} für die Recherche von Unternehmensdaten und-statistiken sowie Unternehmensdatenbanken
\item \textbf{Webseiten von Bundes- und Landesbehörden sowie der EU}
 Statistisches Bundesamt (http://www.destatis.de)
\\Presse- und Informationsamt der Bundesregierung (http://www.bundesregierung.de)
\item \textbf{Webseiten von Marktforschungsinstituten}
(für Marktanteile und Verbraucheranalysen)
\item \textbf{Webseiten von Verbänden und Kammern}
Institut der deutschen Wirtschaft (http://www.deutsche-wirtschaft.de)
\end{itemize}
\end{comment}