%\section{Entwicklung eines konzeptuellen Rahmens}
\section{Vorgehensmodell für Cloud-Migrationen zu Salesforce}
\label{cha:entwickelung_vorgehensmodell}
In diesem Kapitel soll das Fünf-Phasen-Modell angepasst und erweitert werden, 
um den Anforderungen eines Independent Software Vendors (ISV) zu entsprechen. \\

Das Fünf-Phasen-Modell beginnt mit einer technischen und 
wirtschaftlichenMachbarkeitsstudie. Die Migration einer On-Premise-Software in 
die Cloud bedeutet für das migrierende Unternehmen die Erschließung eines ganz 
neuen Marktes, der sich grundlegend vom bekannten Markt unterscheidet. Aus 
diesem Grund kann das Cloud-Produkt sich grundlegend vom bisherigen 
On-Premise-Produkt unterscheiden. Deshalb wird 
\citeflow{how_saas_changes_an_isvs_business} und 
\citeflow{towards_modelling_a_cloud_applications_life_cycle} eine zusätzliche 
Phase vorgeschlagen, in der eine Vision der künftigen Cloud-Lösung entworfen 
wird. Mit dieser Vision kann der Leistungsumfang abgeschätzt werden und auch, 
wie sich das Unternehmen verändern muss, um dieser Vision zu entsprechen. 
(Kapitel~\ref{cha:phaseI})\\

Anschließend kann in Phase II... \\

In Phase III... \\

In Phase IV... \\

In Phase V... \\

In Phase VI... \\

\subsection{Phase I: Vision, Strategie und Organisation}
\label{cha:phaseI}

Als "`disruptive innovation"' ist die Cloud auch ein neuer Markt, auf dem 
qualitativ hochwertige, teilweise hoch spezialisierte IT-Dienstleistungen 
gehandelt werden. 
\pcite{}{}{towards_modelling_a_cloud_applications_life_cycle} \\
Durch das pay-per-use-Preismodell sind auch kleine Unternehmen in der Lage, 
diese Technologien und Dienstleistungen zu nutzen. 
\pcite{}{156}{cloud_migration} Dementsprechend ist die Nutzung der Cloud per se 
weder  innovativ -- da jeder Mitbewerber die Technologie nutzen kann -- noch 
ein Wettbewerbsvorteil: Sie ist ein wirtschaftliches Erfordernis, um den 
Wettbewerb nicht zu verlieren. 
\pcite{}{}{challenges_of_cloud_computing_in_business} 


Damit die in die Cloud migrierte Software dem Unternehmen auch einen anhaltenden 
Wettbewerbsvorteil bieten kann, muss sie einen Wert für den Kunden schaffen, 
unter aktuellen und möglichen zukünftigen Konkurrenten einzigartig oder 
wenigstens selten sein, annähernd unnachahmlich und schwer substituierbar sein.
\pcite{}{}{theoretical_perspectives_for_strategic_human_resource_management} 
\begin{comment}
zufolge
\begin{itemize}
  \item dem Unternehmen einen positiven Wert hinzufügen
  \item unter aktuellen oder möglichen zukünftigen Konkurrenten einzigartig oder 
selten sein
  \item annähernd unnachahmlich
  \item durch Konkurrenten schwer substituierbar sein
\end{itemize}
\end{comment}
Um diesen Anforderungen gerecht zu werden, bedarf es einer geschickten 
Integration standardisierter, in der Cloud verfügbaren Komponenten zu einer 
innovativen Gesamtlösung. Gelingt dies nicht, ist eine Differenzierung von 
konkurrierenden, sich gleichenden Produkten nur über einen niedrigeren Preis 
möglich. \pcite{}{}{towards_modelling_a_cloud_applications_life_cycle} \\

Diese Anforderungen berücksichtigend, sollen -- angelehnt an die Methode der 
SWOT-Analyse \pcite{}{501}{marketingmanagement} -- Chancen und Risiken der 
Cloud betrachtet werden und schließlich in eine Vision. \\

Auch wenn in diesem Modell, als Ergebnis dieser Phase nur die Vision des 
Produktes in den folgenden Phasen weiter verfolgt wird, sollten Chancen und 
Risiken -- ob technischer oder wirtschaftlicher Natur -- auch in Hinblick auf 
ihre Auswirkungen auf das Geschäftsmodell und die 
Unternehmensorganisation des ISV hin untersucht werden. \\

\subsubsection{Geschäftsmodell}
Aus einer technologischen Innovation wird für ein Unternehmen erst ein Wert, 
wenn es mit einem erfolgreichen Geschäftsmodell vermarktet werden kann. Gerade 
beim Auftreten von "`disruptives innovations"' scheitern viele Unternehmen, 
weil sie nicht in der Lage oder willens sind, ihr Geschäftsmodell in 
ausreichendem Maße zu 
ändern. \pcite{}{}{disruptive_technologies_a_business_model_perspective} Um ein 
wirtschaftlich nachhaltiges, an die Realitäten des Marktes angepasstes, 
wettbewerbsfähiges Geschäftsmodell für den Cloud-Markt zu entwickeln, sollte 
der ISV unter Berücksichtigung der Chancen und Risiken die folgenden Fragen 
beantworten \citeflow{disruptive_technologies_a_business_model_perspective}:
\begin{enumerate}
	\item Wie entsteht für den Kunden Wert in Form eines Produktes oder 
		einer Dienstleistung?
	\item In welcher Form und Höhe und mit welchem Preismodell wird Umsatz 
generiert? 
	\item Wie können bereits bestehende und kommende standardisierte 
Komponenten und Dienstleistungen in das Produkt integriert werden?
	\item Wie lassen sich bestehende oder zu erwerbende Ressourcen und 
Fähigkeiten (andersartig) nutzen, um neue Produkte oder Dienstleistungen zu 
erzeugen?
	\item Mit welchen strategischen Entscheidungen lassen sich 
Wettbewerbsvorteile (auch gegenüber On-Premise-Lösungen und 
zugehörigen Preismodellen) erlangen?
\end{enumerate}

\begin{comment}
\subsubsection{Organisationsstruktur}
Da der Kunde des ISV keine Serverinfrastruktur betreiben muss, um die 
Cloud-Lösung zu nutzen und auch auf den Rechnern der Endbenutzer nichts 
installiert werden muss, entfallen im Idealfall Verhandlungen zwischen ISV und 
der IT-Abteilung des Kunden; Entscheidungen werden in kleineren Kreisen, direkt 
von Fachabteilungen getroffen. Für den ISV hat dies zur Folge, dass er mit 

Als Abstraktionsschicht ermöglicht es Cloud-Computing Unternehmen die 
Wertschöpfungskette zu verschlanken und sich auf ihr Kerngeschäft zu 
konzentrieren.

Cloud-Computing wird im Ideal als Abstraktionsschicht gesehen, die 
Komplexitäten 
vor Fachabteilungen und Führungskräften verbirgt und es ihnen ermöglicht ohne 
Entwickler. Wo in der Vergangenheit Empfehlungen, Design, Entwicklung, 
Deployment und Wartung in den Händen von IT-Abteilungen lagen, ist es im 
Cloud-Computing nötig, dass Führungskräfte

Auf dem Weg zum einzigartigen, innovativen und wettbewerbsfähigen Produkt, muss 
sich ein Unternehmen auf seine Kernkompetenzen konzentrieren und 
das Thema IT neu betrachten, um die Flexibilität und Agilität der Cloud nutzen 
zu können. Anstatt sich in der Hauptsache die bestehende 
IT-Infrastruktur zu unterhalten, werden IT-Abteilungen zu strategischen 
Partnern 
in der Weiterentwicklung der Produkte 
\pcite{}{}{how_saas_changes_an_isvs_business}: Mitarbeiter aus der IT müssen 
genutzt werden, um qualitativ hochwertige, nutzbare 
Trends bei Cloud-Dienstleistungen frühzeitig zu erkennen und kreativ in das 
Produkt einfließen zu lassen oder Business-Prozesse bestmöglich zu 
unterstützen.
\end{comment}


\subsubsection{Chancen}


\newcommand{\cloudFeature}[4]{
\item[#1] \hfill \\
\textbf{Beschreibung:} #2 \\
\textbf{Quelle(n):} \pcite{}{}{#3} \\
\textbf{Fragen \& Herausforderungen:} #4
}

\begin{description}
	\cloudFeature{Soziales Element}
	{Das soziale Element entsteht durch neue Möglichkeiten 
des Austausches, die durch die Nutzung einer gemeinsamen Cloud-Plattform 
entstehen: zwischen Nutzern innerhalb eines Unternehmens, zwischen 
verschiedenen Unternehmen oder mit den Entwicklern. Die Weiterentwicklung wird 
dadurch inklusiver, Nutzer lassen sich einbeziehen.}
	{cloud_based_next_generation_service_and_key_challenges, 
changes_in_requirements_engineering}
	{Wie lässt sich der Kontakt zu und zwischen Kunden und ihren 
Fachabteilungen so 
direkt, einfach und produktiv wie möglich gestalten? Lassen sich Communities 
aufbauen?}

	\cloudFeature{Analysemöglichkeiten}{Da sich alle Benutzer auf der Cloud 
bewegen, fallen viel mehr Informationen an, die analysiert werden 
können}{cloud_based_next_generation_service_and_key_challenges,
changes_in_requirements_engineering}{Wie lassen 
sich künftige Entscheidungen mit den gewonnenen Informationen fundierter 
treffen?}

\cloudFeature{Mobilität }{ Lösungen aus SaaS-Bereich sind häufig bereits im 
Standard auf 
mobile Bedienbarkeit 
ausgelegt. }{cloud_based_next_generation_service_and_key_challenges} { 
Um bezüglich Mobilität nicht nur Erwartungen zu erfüllen, sondern 
Begeisterung zu wecken, sollte geprüft werden, wie die gewonnene Mobilität im 
konkreten Fall den Kundenwert steigern kann. }
%%%%%%%%%%%%%%%%%%%%%%%%%%%%%%%%%%%%%%%%%%%%%%%%%%%%%%%%%%%%%%%%%

\cloudFeature{Reduzierte Markteintrittskosten, Skalierte Märkte }{ Durch 
pay-per-use-Modelle 
sind die Markteintrittskosten drastisch reduziert.
}{cloud-computing_the_business_perspective} { Mit welchen Produkten 
lassen sich neue Märkte erschließen? Welche unerschlossenen Märkte gibt es? Wie 
lassen sich geographisch weit entfernte Märkte erschließen? }

%%%%%%%%%%%%%%%%%%%%%%%%%%%%%%%%%%%%%%%%%%%%%%%%%%%%%%%%%%%%%%%%%
\cloudFeature{Skalierung der Leistung }{ In der Cloud stehen -- dynamisch an den 
aktuellen 
Bedarf angepasst -- unbegrenzte Ressourcen bereit.
}{cloud-computing_the_business_perspective} { Wie lassen sich die 
Ressourcen nutzen, um gegenüber On-Premise-Anwendungen im Vorteil zu sein? }
%%%%%%%%%%%%%%%%%%%%%%%%%%%%%%%%%%%%%%%%%%%%%%%%%%%%%%%%%%%%%%%%%

\cloudFeature{Time to market, kürzere Releasezyklen }{ 
Die Software und auch Updates lassen sich schneller auf den Markt bringen. 
}{changes_in_requirements_engineering,
towards_an_understanding_of_cloud_computings_impact_on_org_it_strategy}{}

\cloudFeature{Alternativen testen }{ 
In der Cloud lassen sich alternative Implementierungen 
testen und direkt auswerten.}{changes_in_requirements_engineering} {} 

\cloudFeature{Sparen der Wartung älterer Versionen }{ 
Kapazitäten werden frei, da es in der Cloud nur eine aktuelle Version gibt, in 
die alle Entwicklungsarbeit fließen kann und keine Altversionen gewartet und 
berücksichtigt werden müssen. 
}{changes_in_requirements_engineering,transitioning_to_saas} {}

\cloudFeature{Standardisierte Komponenten}{Mit in der Cloud verfügbaren, 
standardisierten Komponenten lassen und loser Kopplung lassen sich 
Funktionen relativ leicht 
ergänzen}
{towards_an_understanding_of_cloud_computings_impact_on_org_it_strategy,
cloud_based_next_generation_service_and_key_challenges}{}
\end{description}





\subsubsection{Risiken}
\begin{table}[ht!]
\centering
\begin{longtable}{|p{0.11\textwidth}|p{0.4\textwidth}|p{0.4\textwidth}|}
\hline
\textbf{Stichwort} & \textbf{Beschreibung \& Quelle} & \textbf{Fragen \& 
Aufgaben für den ISV} \\
\hline %%%%%%%%%%%%%%%%%%%%%%%%%%%%%%%%%%%%%%%%%%%%%%%%%%%%%%%%%%%%%%%%%

\hline %%%%%%%%%%%%%%%%%%%%%%%%%%%%%%%%%%%%%%%%%%%%%%%%%%%%%%%%%%%%%%%%%
\end{longtable}
\caption{Mögliche Risiken durch die Migration in die Cloud}
\label{tab:risiken_der_cloud}
\end{table}





\begin{comment}

\subsection{Phase I: Business, Strategie und Struktur neu gestalten}

\subsubsection{IT-Strategie}
Unter einer IT-Strategie 
verstehen \citeflow{
towards_an_understanding_of_cloud_computings_impact_on_org_it_strategy} die 
organisationsweite Perspektive auf Investitionen in IT-Systeme sowie das 
Deployment, die Nutzung und das Management von IT-Systemen. Die IT-Strategie 
legt insbesondere fest
\begin{itemize}
	\item welchen Umfang die IT im Unternehmen hat
	\item welche IT-Fähigkeiten vorgehalten werden
	\item wie Steuerung und Controlling erfolgen
	\item wie das Anwendungsportfolio zusammengestellt ist
	\item wie Daten verarbeitet und gespeichert werden
	\item wie IT und Business aufeinander abgestimmt werden
\end{itemize}
\citeflow{
towards_an_understanding_of_cloud_computings_impact_on_org_it_strategy} 
identifizieren einige Auswirkungen einer Cloud-Migrationen auf die IT-Strategie 
für Unternehmen. Aus den allgemein gehaltenen Auswirkungen, werden Vorschläge 
abgeleitet, die sich an Independent Software Vendors richtet.
\begin{description}
  \item[Architektur] Bei klassischen On-Premise-Anwendung war es in der Regel 
nötig, Änderungen äußerst sorgfältig und langfristig zu planen, da sich Fehler 
aufgrund monolithischer Architekturen auf das gesamte System ausgewirkt hätten. 
Die geringe Geschwindigkeit verhindert schnelle Release-Zyklen von 
wenigen Wochen, die in der Cloud erwartet werden. In der Cloud lassen sich neue 
Funktionalitäten durch die lose Kopplung vorgefertigter Komponenten ergänzen. 
\pcite{}{}{cloud_based_next_generation_service_and_key_challenges} Da die 
Komplexität mit zunehmender Zahl von Komponenten -- gerade von verschiedenen 
Anbietern -- trotzdem zunimmt, sollten zukünftige Anforderungen bedacht werden.
  \item[Datenverarbeitung und -Speicherung] Der ISV sollte frühzeitig bedenken, 
welche Daten seiner Kunden in welchem Umfang auf welche Weise in die Cloud 
migriert werden müssen. Inkonsistenzen in den Daten, die auftreten können, wenn 
Daten inkrementell übertragen werden müssen, die Übertragung zeitintensiv ist 
oder wenn On-Premise- und Cloud-Lösung zunächst parallel betrieben werden, 
sind zu berücksichtigen. \\
\end{description}









\begin{comment}
Hauptmotivation die Cloud zu Nutzen, sollte nicht die Reduktion von Kosten
sein, sondern strategische Vorteile, wie eine Konzentration auf das
Kerngeschäft, schnellere und effizientere Innovationsprozesse,
Produktivitätssteigerungen und eine IT, die das Business besser unterstützt,
womöglich sogar profitabel ist. \pcite{}{}{the_strategic_value_of_the_cloud}

Gerade weil die Migration in die Cloud große technische aber vor allem auch
geschäftliche Umwälzungen mit sich bringt, sollte die Migration wirtschaftlich
begründet werden, genauer gesagt:
strategisch. \pcite{}{}{challenges_and_assessment_in_migrating} Auch wenn die
Nutzung der Cloud Einsparungen ermöglicht, machen IT-Budgets in der Regel
nur einen geringen Prozentsatz des Umsatzes aus. Hauptmotivation für die
Migration in die Cloud sollten deshalb strategische Ziele sein, mit denen der
Umsatz ausgebaut oder zumindest behauptet wer
\end{comment}

\begin{comment}
\subsubsection{SWOT-Analyse}
Aus \pcite{}{}{cloud-computing_the_business_perspective}
\begin{description}
	\item[Strengths] \hfill \\
	\begin{itemize}
		\item Skalierbarkeit
		\item 
	\end{itemize}
	\item[Weaknesses] \hfill \\
	\item[Opportunities] \hfill \\
	\item[Threats] \hfill \\
	
\end{description}



\subsection{Phase II: Machbarkeitsstudie}
\subsubsection{Technische Machbarkeit}


\subsubsection{Wirtschaftliche Machbarkeit}
"`In diesem Punkt unterscheidet sich Cloud-Computing von früheren Paradigmen
wie Outsourcing, welches nicht auf das Geschäftsmodell des Unternehmens wirken
will. Durch neue Anwendungsszenarien kann mit Cloud-Computing ein beachtlicher
Mehrwert geschaffen werden."' \pcite{}{154}{cloud_migration}

"`Cloud-Computing verändert die Preisgestaltung von IT-Services und kann als
eine Reaktion  auf das straffere Kostenmanagement im Nachgang der Finanzkrise
2008 verwendet werden. Unternehmenskunden wollen sich nicht mehr mit fixen
IT-Kosten binden. Sie wollen auch in der IT des Unternehmens die Capital
Expenditures (CAPEX) in Operational Expenditures (OPEX) umwandeln. Dies ist der
Hauptgrund, wieso das 'Pay-per-Use-Pricing' von Cloud-Services bei den
 Unternehmen so viel Interesse geweckt hat."' \pcite{}{156}{cloud_migration}


"`Mögliche betriebswirtschaftliche Vorteile: bessere Nutzung des
Ressourcen-Pools des Unternehmens, Verbesserung der Zeit zur Markteinführung
neuer Produkte und Dienstleistungen, Steigerung der Agilität und letztlich
Verbesserung der Erfahrungen der Kunden im Umgang mit dem Unternehmen,
welches auf Cloud-Computing setzt."'
\pcite{}{157}{cloud_migration}


"`Alle Empfehlungen sollten mit einem Business Case hinterlegt werden, der
die Höhe der Kostenreduktion und die Verbesserung des Servicelevels zeigt."'
\pcite{}{158}{cloud_migration}

\subsection{Phase III: Anforderungsanalyse und -Planung}
\subsubsection{Anforderungsermittlung}
\subsubsection{Return on Investment (ROI)}
\subsubsection{Total Cost of Ownership (TCO)}

\subsection{Phase IV: Migration}
\subsection{Phase V: Tests und Auslieferung}
\subsection{Phase VI: Überwachung und Wartung}
\subsection{Gesamtbetrachtung}
\subsubsection{Agilität}



\begin{comment}
\label{cha:entwicklung}
Dieses Kapitel dient der Entwicklung eines konzeptuellen Rahmens auf Basis theoretischer Grundlagen, vorausgesetzt sie verfolgen einen positivistischen Ansatz. Hierfür leiten Sie Hypothesen aus verschiedenen sinnvoll kombinierten Quellen her. Hierdurch generieren Sie aus bestehendem Wissen neues Wissen, was eine Eigenleistung und somit ein wichtiger Bestandteil Ihrer Arbeit darstellt.

Sollte Ihre Arbeit nicht positivistisch ausgelegt sein, stellt dieser Abschnitt kein Pflichtkapitel der Arbeit dar. Alternativ beschreiben Sie Anforderungen für ein mögliches Konzept oder verzichten vollständig auf dieses Kapitel.

\textbf{Setzen Sie sich frühzeitig mit Ihrem Betreuer in Verbindung, um Ihre Gliederung abzustimmen und mögliche Missverständnisse zu beseitigen.}

Im Folgenden werden einige allgemeine Hinweise zu den Themen richtiges Zitieren und Literaturrecherche gegeben.


\subsection{Quellen und richtiges Zitieren}
Quellen können in Fußnote oder direkt im Text platziert werden. Alles was nicht Ihr eigenes Gedankengut darstellt, muss mit einer entsprechenden Quelle belegt werden. Hierbei können wörtliche und indirekte Zitate verwendet werden. Wörtliche Zitate sind immer mit der Seitennummer der Quelle anzugeben.

Beispiel für ein direktes Zitat:

%\textit{\glqq The case study is a research strategy which focuses in understanding the dynamics present within single settings\grqq} \pcite{}{543}{eisenhardt1989}.

Beispiel für ein indirektes Zitat:

%Eine explorative Fallstudie dient der Gewinnung von neuen Erkenntnissen und der Bildung von neuen Hypothesen über bestimmte Sachverhalte. Durch den Beitrag zum Theorieaufbau ist der Erkenntnisgewinn höher als bei einer reinen deskriptiven Fallstudie. In explorativen Fallstudien werden Phänomene in noch wenig erforschten Gebieten identifiziert und aus erkannten Zusammenhängen neue Hypothesen gebildet \citepara{eisenhardt1989}.

Alternativ kann die Quelle auch im  laufenden Text angegeben werden:

%Nach \citeflow{eisenhardt1989} wird die Wichtigkeit der Fallauswahl oft unterschätzt. Die Fälle können zwar zufällig ausgewählt werden, dies ist aber weder notwendig noch wünschenswert.

Quellenangaben bestehen aus Autor, Jahr und ggf. Seitenangabe. Bei zwei Autoren sind beide Autoren zu nennen, bei mehreren Autoren nur der erste Autor mit dem Zusatz „et al.“.\newpage


\subsection{Zitieren mit Endnoten}
Im Rahmen der Erstellung von Arbeiten am Fachgebiet ISE ist das Literaturverwaltungsprogramm EndNote zu verwenden. Dieses steht auf der \href{http://www.ulb.tu-darmstadt.de/service/literaturverwaltung_start/endnote_ulb/endnote.de.jsp
}{ULB-Seite} zum Download verfügbar.


\subsubsection{Lateinischer Text mit Zitaten für Erstellung des Literaturverzeichnisses}
\label{cha:source:latintext}
%Lorem ipsum dolor sit amet, consectetur adipiscing elit. Sed vitae lacus eu
%augue semper lobortis vitae aliquet leo. Fusce eleifend sodales commodo
%\citepara{eisenhardt1989}. Mauris arcu metus, bibendum sagittis condimentum
%eget, placerat a enim \citepara{baechle2010}. Quisque sit amet sagittis lectus.
%Curabitur sit amet libero eu felis elementum mollis. Nullam odio diam, mollis
%vitae viverra ut, laoreet ut odio. Praesent facilisis suscipit consequat. Morbi
%feugiat rutrum erat, eu sagittis nibh rhoncus nec \citepara{melao2000}.

%In euismod, arcu ut semper adipiscing, nibh odio ullamcorper arcu, ut scelerisque massa magna nec quam \citepara{benlian2013}. Curabitur bibendum nibh eget augue pellentesque iaculis \citepara{sheffi2005}. Praesent iaculis auctor gravida. Quisque congue, magna ut bibendum semper, enim tortor ultrices lorem, ac feugiat tortor lectus nec nunc \citepara{carnap1974}. Pellentesque habitant morbi tristique senectus et netus et malesuada fames ac turpis egestas. Lorem ipsum dolor sit amet, consectetur adipiscing elit. Fusce dignissim, augue a sodales tristique, neque dui mollis arcu, id interdum augue justo sed lacus \citepara{welchering2013}. Vestibulum ante ipsum primis in faucibus orci luctus et ultrices posuere cubilia Curae; Mauris euismod bibendum nulla, sed accumsan urna tempor sed. Etiam eget diam eros, sed aliquet dolor \citepara{broadbent1996}. Phasellus vitae quam in orci convallis pharetra. Donec sit amet imperdiet nisi \citepara{kayser2013}. Sed vel interdum orci. Praesent vulputate, dolor id varius egestas, enim libero cursus neque, a cursus sapien nulla ut augue. Nullam vitae tortor nisl, vitae cursus enim. Suspendisse eget metus ipsum, sit amet varius sem \citepara{shazly2013}.

\subsection{Literaturrecherche}
Anbei eine kurze Auflistung von möglichen Kanälen zur Literaturrecherche.

\textbf{Zu Verwaltung Ihrer Literatur benutzen Sie bitte das Programm EndNote, dieses wird kostenfrei von der TU zu Verfügung gestellt.}

\url{http://www.ulb.tu-darmstadt.de/angebot/service/literaturverwaltung/endnote.de.jsp}

\subsubsection{Angebot der ULB}
\begin{itemize}
\item Universitätsbibliotheken (\url{http://www.ulb.tu-darmstadt.de/})
\item Rechercheangebot der ULB (\url{http://www.ulb.tu-darmstadt.de/recherche/})
\end{itemize}

\subsubsection{Online-Datenbanken und -Bibliotheken}
\begin{itemize}
\item Elektronische Zeitschriftenbibliothek (EZB) \\
(\url{http://rzblx1.uni-regensburg.de/ezeit/fl.phtml?bibid=TUDA})
\item AIS Electronic Library (AISeL)\\
(\url{http://aisel.aisnet.org/})
\item Zeitschriftendatenbank (ZDB)\\
(\url{http://dispatch.opac.ddb.de/DB=1.1/srt=YOP/})
\item Datenbank-Infosystem (DBIS): Literatur- und Fakten-Datenbank\\
(\url{http://rzblx10.uni-regensburg.de/dbinfo/fachliste.php?bib_id=tud})
\item IEEE Xplore \\
(\url{http://ieeexplore.ieee.org/Xplore/dynhome.jsp?tag=1})
\item EBSCO: internationale wirtschafts-wiss. Zeitschriften\\ (\url{http://search.ebscohost.com})
\item Springer-Online: Bücher/Beiträge des Springer Verlags\\
(\url{http://www.springerlink.com})
\item WiSo Net: deutschsprachige Literatur zu Wirtschafts- und Sozialwissenschaften\\
(\url{www.wiso-net.de})

\end{itemize}

\subsubsection{Sonstiges}
\begin{itemize}
\item \textbf{Google Scholar:} Suchdienst für wissenschaftliche Recherchen (http://scholar.google.de)
\item \textbf{Verlagswebseiten} Recherche und den Zugriff auf Zeitschriften- und Zeitungsartikel und E-Books
\item \textbf{Webseiten von Unternehmen} für die Recherche von Unternehmensdaten und-statistiken sowie Unternehmensdatenbanken
\item \textbf{Webseiten von Bundes- und Landesbehörden sowie der EU}
 Statistisches Bundesamt (http://www.destatis.de)
\\Presse- und Informationsamt der Bundesregierung (http://www.bundesregierung.de)
\item \textbf{Webseiten von Marktforschungsinstituten}
(für Marktanteile und Verbraucheranalysen)
\item \textbf{Webseiten von Verbänden und Kammern}
Institut der deutschen Wirtschaft (http://www.deutsche-wirtschaft.de)
\end{itemize}
\end{comment}
