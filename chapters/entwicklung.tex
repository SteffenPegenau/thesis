%\section{Entwicklung eines konzeptuellen Rahmens}
\section{Vorgehensmodell für Cloud-Migrationen zu Salesforce}
\label{cha:entwickelung_vorgehensmodell}
In diesem Kapitel soll das Fünf-Phasen-Modell angepasst und erweitert werden, 
um den Anforderungen eines Independent Software Vendors (ISV) zu entsprechen. \\

Das Fünf-Phasen-Modell beginnt mit einer technischen und 
wirtschaftlichenMachbarkeitsstudie. Die Migration einer On-Premise-Software in 
die Cloud bedeutet für das migrierende Unternehmen die Erschließung eines ganz 
neuen Marktes, der sich grundlegend vom bekannten Markt unterscheidet. Aus 
diesem Grund kann das Cloud-Produkt sich grundlegend vom bisherigen 
On-Premise-Produkt unterscheiden. Deshalb wird 
\citeflow{how_saas_changes_an_isvs_business} und 
\citeflow{towards_modelling_a_cloud_applications_life_cycle} eine zusätzliche 
Phase vorgeschlagen, in der eine Vision der künftigen Cloud-Lösung entworfen 
wird. Mit dieser Vision kann der Leistungsumfang abgeschätzt werden und auch, 
wie sich das Unternehmen verändern muss, um dieser Vision zu entsprechen. 
(Kapitel~\ref{cha:phaseI})\\

Anschließend kann in Phase II... \\

In Phase III... \\

In Phase IV... \\

In Phase V... \\

In Phase VI... \\

\subsection{Phase I: Entwicklung einer Vision}
\label{cha:phaseI}

Als "`disruptive innovation"' ist die Cloud auch ein neuer Markt, auf dem 
qualitativ hochwertige, teilweise hoch spezialisierte IT-Dienstleistungen 
gehandelt werden. 
\pcite{}{}{towards_modelling_a_cloud_applications_life_cycle} \\
Durch das pay-per-use-Preismodell sind auch kleine Unternehmen in der Lage, 
diese Technologien und Dienstleistungen zu nutzen. 
\pcite{}{156}{cloud_migration} Dementsprechend ist die Nutzung der Cloud per se 
weder  innovativ -- da jeder Mitbewerber die Technologie nutzen kann -- noch 
ein Wettbewerbsvorteil: Sie ist ein wirtschaftliches Erfordernis, um den 
Wettbewerb nicht zu verlieren. 
\pcite{}{}{challenges_of_cloud_computing_in_business} 


Damit die in die Cloud migrierte Software dem Unternehmen auch einen anhaltenden 
Wettbewerbsvorteil bieten kann, muss sie einen Wert für den Kunden schaffen, 
unter aktuellen und möglichen zukünftigen Konkurrenten einzigartig oder 
wenigstens selten sein, annähernd unnachahmlich und schwer substituierbar sein.
\pcite{}{}{theoretical_perspectives_for_strategic_human_resource_management} 
\begin{comment}
zufolge
\begin{itemize}
  \item dem Unternehmen einen positiven Wert hinzufügen
  \item unter aktuellen oder möglichen zukünftigen Konkurrenten einzigartig oder 
selten sein
  \item annähernd unnachahmlich
  \item durch Konkurrenten schwer substituierbar sein
\end{itemize}
\end{comment}
Um diesen Anforderungen gerecht zu werden, bedarf es einer geschickten 
Integration standardisierter, in der Cloud verfügbaren Komponenten zu einer 
innovativen Gesamtlösung. Gelingt dies nicht, ist eine Differenzierung von 
konkurrierenden, sich gleichenden Produkten nur über einen niedrigeren Preis 
möglich. \pcite{}{}{towards_modelling_a_cloud_applications_life_cycle} \\

Diese Anforderungen berücksichtigend, sollen -- angelehnt an die Methode der 
SWOT-Analyse \pcite{}{501}{marketingmanagement} -- Chancen und Risiken der 
Cloud betrachtet werden und schließlich in eine Vision. \\

Auch wenn in diesem Modell, als Ergebnis dieser Phase nur die Vision des 
Produktes in den folgenden Phasen weiter verfolgt wird, sollten Chancen und 
Risiken -- ob technischer oder wirtschaftlicher Natur -- auch in Hinblick auf 
ihre Auswirkungen auf das Geschäftsmodell und die 
Unternehmensorganisation des ISV hin untersucht werden. \\

\subsubsection{Geschäftsmodell}
Aus einer technologischen Innovation wird für ein Unternehmen erst ein Wert, 
wenn es mit einem erfolgreichen Geschäftsmodell vermarktet werden kann. Gerade 
beim Auftreten von "`disruptives innovations"' scheitern viele Unternehmen, 
weil sie nicht in der Lage oder willens sind, ihr Geschäftsmodell in 
ausreichendem Maße zu 
ändern. \pcite{}{}{disruptive_technologies_a_business_model_perspective} Um ein 
wirtschaftlich nachhaltiges, an die Realitäten des Marktes angepasstes, 
wettbewerbsfähiges Geschäftsmodell für den Cloud-Markt zu entwickeln, sollte 
der ISV unter Berücksichtigung der Chancen und Risiken die folgenden Fragen 
beantworten \citeflow{disruptive_technologies_a_business_model_perspective}:
\begin{enumerate}
	\item Wie entsteht für den Kunden Wert in Form eines Produktes oder 
		einer Dienstleistung?
	\item In welcher Form und Höhe und mit welchem Preismodell wird Umsatz 
generiert? 
	\item Wie können bereits bestehende und kommende standardisierte 
Komponenten und Dienstleistungen in das Produkt integriert werden?
	\item Wie lassen sich bestehende oder zu erwerbende Ressourcen und 
Fähigkeiten (andersartig) nutzen, um neue Produkte oder Dienstleistungen zu 
erzeugen?
	\item Mit welchen strategischen Entscheidungen lassen sich 
Wettbewerbsvorteile (auch gegenüber On-Premise-Lösungen und 
zugehörigen Preismodellen) erlangen?
\end{enumerate}

\begin{comment}
\subsubsection{Organisationsstruktur}
Da der Kunde des ISV keine Serverinfrastruktur betreiben muss, um die 
Cloud-Lösung zu nutzen und auch auf den Rechnern der Endbenutzer nichts 
installiert werden muss, entfallen im Idealfall Verhandlungen zwischen ISV und 
der IT-Abteilung des Kunden; Entscheidungen werden in kleineren Kreisen, direkt 
von Fachabteilungen getroffen. Für den ISV hat dies zur Folge, dass er mit 

Als Abstraktionsschicht ermöglicht es Cloud-Computing Unternehmen die 
Wertschöpfungskette zu verschlanken und sich auf ihr Kerngeschäft zu 
konzentrieren.

Cloud-Computing wird im Ideal als Abstraktionsschicht gesehen, die 
Komplexitäten 
vor Fachabteilungen und Führungskräften verbirgt und es ihnen ermöglicht ohne 
Entwickler. Wo in der Vergangenheit Empfehlungen, Design, Entwicklung, 
Deployment und Wartung in den Händen von IT-Abteilungen lagen, ist es im 
Cloud-Computing nötig, dass Führungskräfte

Auf dem Weg zum einzigartigen, innovativen und wettbewerbsfähigen Produkt, muss 
sich ein Unternehmen auf seine Kernkompetenzen konzentrieren und 
das Thema IT neu betrachten, um die Flexibilität und Agilität der Cloud nutzen 
zu können. Anstatt sich in der Hauptsache die bestehende 
IT-Infrastruktur zu unterhalten, werden IT-Abteilungen zu strategischen 
Partnern 
in der Weiterentwicklung der Produkte 
\pcite{}{}{how_saas_changes_an_isvs_business}: Mitarbeiter aus der IT müssen 
genutzt werden, um qualitativ hochwertige, nutzbare 
Trends bei Cloud-Dienstleistungen frühzeitig zu erkennen und kreativ in das 
Produkt einfließen zu lassen oder Business-Prozesse bestmöglich zu 
unterstützen.
\end{comment}


\subsubsection{Chancen \& Risiken der Cloud-Migration}
\label{cha:chancen_und_risken}
Chancen und Risiken einer Cloud-Migration für einen Independent Software 
Vendor sind Ergebnis der Literaturrecherche und finden sich in den
Tabellen~\ref{tab:chancen_der_cloud} und \ref{tab:risiken_der_cloud}. Die 
Spalte "`Fragen \& Aufgaben für den ISV"' soll dabei helfen, eine 
Produktvision zu entwickeln, die dem Unternehmen einen anhaltenden 
Wettbewerbsvorteil bescheren soll.
\begin{table}[ht!]
\centering
\begin{longtable}{|p{0.15\textwidth}|p{0.38\textwidth}|p{0.38\textwidth}|}
\hline
\textbf{Bezeichnung} & \textbf{Beschreibung \& Quelle} & \textbf{Fragen \& 
Aufgaben für den ISV} \\
\hline %%%%%%%%%%%%%%%%%%%%%%%%%%%%%%%%%%%%%%%%%%%%%%%%%%%%%%%%%%%%%%%%%
Soziales Element & Das soziale Element entsteht durch neue Möglichkeiten 
des Austausches, die durch die Nutzung einer gemeinsamen Cloud-Plattform 
entstehen: zwischen Nutzern innerhalb eines Unternehmens, zwischen 
verschiedenen Unternehmen oder mit den Entwicklern. Die Weiterentwicklung wird 
dadurch inklusiver, Nutzer lassen sich einbeziehen. 
\pcite{}{}{cloud_based_next_generation_service_and_key_challenges, 
changes_in_requirements_engineering} &
Wie lässt sich der Kontakt zu und zwischen Kunden und ihren Fachabteilungen so 
direkt, einfach und produktiv wie möglich gestalten? Lassen sich Communities 
aufbauen? \\
\hline %%%%%%%%%%%%%%%%%%%%%%%%%%%%%%%%%%%%%%%%%%%%%%%%%%%%%%%%%%%%%%%%%
Analyse\-möglich\-keiten & Da sich alle Benutzer auf der Cloud 
bewegen, fallen viel mehr Informationen an, die analysiert werden können
\pcite{}{}{cloud_based_next_generation_service_and_key_challenges,
changes_in_requirements_engineering} & Wie lassen 
sich künftige Entscheidungen mit den gewonnenen Informationen fundierter 
treffen? \\
\hline %%%%%%%%%%%%%%%%%%%%%%%%%%%%%%%%%%%%%%%%%%%%%%%%%%%%%%%%%%%%%%%%%
Mobilität & Lösungen aus SaaS-Bereich sind häufig bereits im Standard auf 
mobile Bedienbarkeit 
ausgelegt. \pcite{}{}{cloud_based_next_generation_service_and_key_challenges} & 
Um bezüglich Mobilität nicht nur Erwartungen zu erfüllen, sondern 
Begeisterung zu wecken, sollte geprüft werden, wie die gewonnene Mobilität im 
konkreten Fall den Kundenwert steigern kann. \\
\hline %%%%%%%%%%%%%%%%%%%%%%%%%%%%%%%%%%%%%%%%%%%%%%%%%%%%%%%%%%%%%%%%%
Reduzierte Markteintrittskosten \& Skalierte Märkte & Durch pay-per-use-Modelle 
sind die Markteintrittskosten drastisch reduziert.
\pcite{}{}{cloud-computing_the_business_perspective} & Mit welchen Produkten 
lassen sich neue Märkte erschließen? Welche unerschlossenen Märkte gibt es? Wie 
lassen sich geographisch weit entfernte Märkte erschließen? \\
\hline %%%%%%%%%%%%%%%%%%%%%%%%%%%%%%%%%%%%%%%%%%%%%%%%%%%%%%%%%%%%%%%%%
Skalierung der Leistung & In der Cloud stehen -- dynamisch an den aktuellen 
Bedarf angepasst -- unbegrenzte Ressourcen bereit.
\pcite{}{}{cloud-computing_the_business_perspective} & Wie lassen sich die 
Ressourcen nutzen, um gegenüber On-Premise-Anwendungen im Vorteil zu sein? \\
\hline %%%%%%%%%%%%%%%%%%%%%%%%%%%%%%%%%%%%%%%%%%%%%%%%%%%%%%%%%%%%%%%%%

Time to market \& kürzere Releasezyklen & 
Die Software und auch Updates lassen sich schneller auf den Markt bringen. 
\pcite{}{}{changes_in_requirements_engineering} &
\\
\hline

Alternativen testen & 
In der Cloud lassen sich alternative Implementierungen 
testen und direkt auswerten. \pcite{}{}{changes_in_requirements_engineering} &
\\ \hline

Sparen der Wartung älterer Versionen & 
Kapazitäten werden frei, da es in der Cloud nur eine aktuelle Version gibt, in 
die alle Entwicklungsarbeit fließen kann und keine Altversionen gewartet und 
berücksichtigt werden müssen. \pcite{}{}{changes_in_requirements_engineering, 
transitioning_to_saas} &
\\ \hline
\end{longtable}
\caption{Chancen durch die Migration in die Cloud}
\label{tab:chancen_der_cloud}
\end{table}
\addtocounter{table}{-1}


\begin{table}[ht!]
\centering
\begin{longtable}{|p{0.11\textwidth}|p{0.4\textwidth}|p{0.4\textwidth}|}
\hline
\textbf{Stichwort} & \textbf{Beschreibung \& Quelle} & \textbf{Fragen \& 
Aufgaben für den ISV} \\
\hline %%%%%%%%%%%%%%%%%%%%%%%%%%%%%%%%%%%%%%%%%%%%%%%%%%%%%%%%%%%%%%%%%

\hline %%%%%%%%%%%%%%%%%%%%%%%%%%%%%%%%%%%%%%%%%%%%%%%%%%%%%%%%%%%%%%%%%
\end{longtable}
\caption{Mögliche Risiken durch die Migration in die Cloud}
\label{tab:risiken_der_cloud}
\end{table}

\subsubsection{Produktvision}
\citeflow{fivephases} schlagen ihr Fünf-Phasen-Modell für eine vollständige 
Migration einer Software und der Gesamtheit ihrer Funktionalität vor. Von 
dieser Vorgabe wird in dem hier vorgestellten Modell aus zweierlei Gründen 
abgewichen. \\
Erstens wird es mit einer 1:1-Migration weder möglich sein die in 
Kapitel~\ref{cha:chancen_und_risken} genannten Vorteile zu realisieren, noch 
die Risiken zu umschiffen. \\
Eine vollständige Migration und eine erst daran anschließende Vermarktung würde 
viel Zeit in Anspruch nehmen, in der keine Umsätze generiert werden. Dadurch 
steigt zweitens das finanzielle Risiko und das Unternehmen läuft Gefahr das 
gesetzte Cloud-Ziel von mehr Flexibilität, höherer Agilität und schnellere 
Innovationsgeschwindigkeit zu verfehlen. Denn dabei würde ein Vorteil der 
Cloud vollkommen ignoriert: Die Software lässt sich leicht aktualisieren, 
während der Kunde bereits mit ihr arbeitet. Dies ermöglicht eine Agilität, wie 
sie nicht möglich ist, wenn Software erst ausgeliefert und vom Kunde installiert 
werden muss. Es wäre denkbar, dass Erfahrungen des Kunden im Umgang mit dem 
jeweils aktuellen Stand der Software in die Sprint-Planung eingehen. Die 
Weiterentwicklung der Software wäre damit eng an die tatsächlichen Bedürfnisse 
der Kunden gekoppelt. \\

Aus den genannten Gründen wird in dieser Arbeit sprintweise Migration und 
Weiterentwicklung vorgeschlagen, die mit einer Produktvision beginnt. 
Abgeleitet aus den Eigenschaften, die Unternehmensvisionen laut 
\citeflow{the_power_of_vision} aufweisen sollten, werden die folgenden Merkmale 
für eine Produktvision vorgeschlagen:
\begin{itemize}
	\item Sie sollte die genannten Chancen und Risiken berücksichtigen.
	\item Der Umfang des Produktes sollte so klein wie möglich, so 
umfangreich wie nötig sein, um frühestmöglich einen auslieferbaren Wert zu 
schaffen. 
	\item Die Architektur sollte -- in Hinblick auf zusätzliche 
Cloud-Komponenten, Dienstleistungen und das Datenmodell -- erweiterbar sein.
\end{itemize} 

Aus \citeflow{the_power_of_vision}:
\begin{itemize}
	\item Vision ist strategische Überlegung um wettbewerbsfähiges Produkt 
zu schaffen
	\item Unterscheidung nötig zwischen starken und schwachen Visionen, 
positiven und negativen
	\item Eigenschaften guter Visionen: \\
	\begin{itemize}
		\item conciseness - kurz
		\item clarity - prime goal, easy (in 5 minutes) to understand
		\item future orientation - long term
		\item stability - Vision reagiert nicht auf kurzfristige Trends
		\item challenge - high but achievable difficulty
		\item abstractness
		\item desirability or ability to inspire
	\end{itemize}
	\item Notwendigkeiten um Vision zu realisieren: \\
	\begin{itemize}
		\item communicating the vision
		\item aligning organizational processes and systems to suit the 
vision
		\item empowering others to act to achieve the vision
		\item motivating staff
	\end{itemize}
	\item Charakterisitika eng mit Unternehmenserfolg verbunden


\end{itemize}

\subsubsection{TODO: Preismodell, frühe Vermarktung eines Produktes mit 
geringerem Umfang}

\begin{comment}

\subsection{Phase I: Business, Strategie und Struktur neu gestalten}

\subsubsection{IT-Strategie}
Unter einer IT-Strategie 
verstehen \citeflow{
towards_an_understanding_of_cloud_computings_impact_on_org_it_strategy} die 
organisationsweite Perspektive auf Investitionen in IT-Systeme sowie das 
Deployment, die Nutzung und das Management von IT-Systemen. Die IT-Strategie 
legt insbesondere fest
\begin{itemize}
	\item welchen Umfang die IT im Unternehmen hat
	\item welche IT-Fähigkeiten vorgehalten werden
	\item wie Steuerung und Controlling erfolgen
	\item wie das Anwendungsportfolio zusammengestellt ist
	\item wie Daten verarbeitet und gespeichert werden
	\item wie IT und Business aufeinander abgestimmt werden
\end{itemize}
\citeflow{
towards_an_understanding_of_cloud_computings_impact_on_org_it_strategy} 
identifizieren einige Auswirkungen einer Cloud-Migrationen auf die IT-Strategie 
für Unternehmen. Aus den allgemein gehaltenen Auswirkungen, werden Vorschläge 
abgeleitet, die sich an Independent Software Vendors richtet.
\begin{description}
  \item[Architektur] Bei klassischen On-Premise-Anwendung war es in der Regel 
nötig, Änderungen äußerst sorgfältig und langfristig zu planen, da sich Fehler 
aufgrund monolithischer Architekturen auf das gesamte System ausgewirkt hätten. 
Die geringe Geschwindigkeit verhindert schnelle Release-Zyklen von 
wenigen Wochen, die in der Cloud erwartet werden. In der Cloud lassen sich neue 
Funktionalitäten durch die lose Kopplung vorgefertigter Komponenten ergänzen. 
\pcite{}{}{cloud_based_next_generation_service_and_key_challenges} Da die 
Komplexität mit zunehmender Zahl von Komponenten -- gerade von verschiedenen 
Anbietern -- trotzdem zunimmt, sollten zukünftige Anforderungen bedacht werden.
  \item[Datenverarbeitung und -Speicherung] Der ISV sollte frühzeitig bedenken, 
welche Daten seiner Kunden in welchem Umfang auf welche Weise in die Cloud 
migriert werden müssen. Inkonsistenzen in den Daten, die auftreten können, wenn 
Daten inkrementell übertragen werden müssen, die Übertragung zeitintensiv ist 
oder wenn On-Premise- und Cloud-Lösung zunächst parallel betrieben werden, 
sind zu berücksichtigen. \\
\end{description}









\begin{comment}
Hauptmotivation die Cloud zu Nutzen, sollte nicht die Reduktion von Kosten
sein, sondern strategische Vorteile, wie eine Konzentration auf das
Kerngeschäft, schnellere und effizientere Innovationsprozesse,
Produktivitätssteigerungen und eine IT, die das Business besser unterstützt,
womöglich sogar profitabel ist. \pcite{}{}{the_strategic_value_of_the_cloud}

Gerade weil die Migration in die Cloud große technische aber vor allem auch
geschäftliche Umwälzungen mit sich bringt, sollte die Migration wirtschaftlich
begründet werden, genauer gesagt:
strategisch. \pcite{}{}{challenges_and_assessment_in_migrating} Auch wenn die
Nutzung der Cloud Einsparungen ermöglicht, machen IT-Budgets in der Regel
nur einen geringen Prozentsatz des Umsatzes aus. Hauptmotivation für die
Migration in die Cloud sollten deshalb strategische Ziele sein, mit denen der
Umsatz ausgebaut oder zumindest behauptet wer
\end{comment}

\begin{comment}
\subsubsection{SWOT-Analyse}
Aus \pcite{}{}{cloud-computing_the_business_perspective}
\begin{description}
	\item[Strengths] \hfill \\
	\begin{itemize}
		\item Skalierbarkeit
		\item 
	\end{itemize}
	\item[Weaknesses] \hfill \\
	\item[Opportunities] \hfill \\
	\item[Threats] \hfill \\
	
\end{description}
\end{comment}



\subsection{Phase II: Machbarkeitsstudie}
Phase II entspricht weitgehend der ersten Phase aus \citeflow{fivephases}: 
Durchführung einer technischen und wirtschaftlichen Machbarkeitsstudie. Objekt 
der beiden Machbarkeitsstudien ist hier jedoch die Produktvision, nicht die 
On-Premise-Software.\\
Dies wirkt sich vor allem auf die Prüfung der Wirtschaftlichkeit aus. Der 
Vorschlag des Fünf-Phasen-Modells an dieser Stelle eine "`detaillierte 
Kosten-Nutzen-Analyse"' durchzuführen, wird hier nicht 
befolgt. Um diese Entscheidung zu begründen, sollen die beiden 
Migrationsansätze noch einmal verglichen werden.\\
\usetikzlibrary{arrows}
\begin{figure}[bh]
\begin{center}
\begin{tikzpicture}
\newcommand{\rechts}{6}
\newcommand{\radius}{1.9cm}
\coordinate (centerTop) at (\rechts,3.5);
\coordinate (centerLinks) at (0,0);
\coordinate (centerRechts) at (\rechts,0);
\coordinate (centerBottom) at (\rechts,-1);

% Linker Kreis
\draw (centerLinks) circle (\radius) node [text=black] 
{On-Premise-Lösung};


% Oberer Kreis
\draw[fill=green] (centerTop) circle (\radius);
\node[align=left] at ($ (centerTop) + (3.5,0)$) {Chancen};

%Unterer Kreis mit Schrift
\draw[fill=red] (centerBottom) circle (\radius);
\node[align=left] at ($ (centerBottom) + (3.5,-1)$) {Risiken};


% Rechte Mitte
\draw (centerRechts) circle (\radius);
\node[align=left] at ($ (centerRechts) + (3.5,0)$) 
{\st{On-Premise-}\\Cloud-Lösung};

% Pfeil
\path[draw=black,solid,line width=1mm,fill=black,
preaction={-triangle 90,thin,draw,shorten >=-1mm}
] ($ (centerLinks) + (1.2 * \radius,0) $) -- ($ (centerRechts) + 
(-1.2 * \radius,0) $);

%\draw \secondcircle node [text=black,left] {On-Premise-Software};
%\draw \thirdcircle node [text=black,right] {$C$};
\end{tikzpicture}
\caption{Funktionsumfang vor und nach der Migration im Wasserfallmodell. 
Selbsterstellte Grafik.}
\label{fig:funktionsumfang_wasserfallmodell}
\end{center}
\end{figure}

In Abbildung~\ref{fig:funktionsumfang_wasserfallmodell} ist der Ansatz aus dem 
Fünf-Phasen-Modell dargestellt. Dabei wird die bestehende Anwendung in ihrem 
gesamten Umfang in die PaaS/SaaS-Cloud migriert und erst daran anschließend 
vermarktet. Zumindest die Chance, eine agile Cloud-Lösung zu schaffen, die 
bereits früh einen Wert für Kunden schafft und sich schnell und an 
Kundenwünschen orientiert weiterentwickelt, wird verschenkt. Der ISV 
läuft aber auch Gefahr weitere Vorteile nicht zu realisieren, wenn er bei der 
Migration zunächst die Herstellung aller Funktionalitäten der Altsoftware in 
den Vordergrund stellt und dabei vernachlässigt, über Möglichkeiten 
nachzudenken, die Vorteile und Chancen der Cloud umzusetzen. Deshalb hat die 
Menge der Chancen nur zu einem geringen Teil in der Cloud-Lösung enthalten -- 
im Gegensatz zu den Risiken, die Teil der Cloud-Lösung werden, wenn man sich 
ihrer nicht bewusst ist und sie daher nicht vermeiden kann. \\
Abbildung~\ref{fig:funktionsumfang_agil} zeigt die in dieser Arbeit 
vorgeschlagene Vorgehensweise. Hier ist die Cloud-Lösung in ihrem Umfang sehr 
viel kleiner als die bestehende On-Premise-Software, sodass sie gerade das 
enthält, was für die Vermarktung nötig ist. Die Beschränkung des Umfangs 
begünstigt eine schnelle Fertigstellung und Vermarktung. Fehlendes Features 
werden agil nachentwickelt und in kurzen Abständen freigegeben. In der 
Cloud-Lösung werden von Anfang an Chancen und Risiken berücksichtigt und 
maximiert beziehungsweise minimiert. \\

Im Ergebnis ist das hier vorgestellte Vorgehen weniger riskant, da nötige 
Investitionen geringer sind und Fehler wie Erfolge schneller sichtbar werden. 
Da in diesem Modell zudem nicht die Kosteneinsparungen sondern strategische 
Ziele im Vordergrund stehen und der zeitliche Horizont viel kleiner ist, kann 
die Wirtschaftlichkeitsprüfung weniger detailliert ausfallen. Es sollte geprüft 
geprüft werden, ob nötige Umsätze mit dem Funktionsumfang der Vision generiert 
werden können. \\

\usetikzlibrary{arrows}
\begin{figure}[h]
\begin{center}

\tikzset{
    %Define standard arrow tip
    >=stealth',
    %Define style for boxes
    punkt/.style={
           rectangle,
           %rounded corners,
           draw=black, very thick,
           text width=10em,
           minimum height=10em,
           text centered},
    % Define arrow style
    pil/.style={
           ->,
           very thick,
           shorten <=5pt,
           shorten >=5pt,},
    sepLine/.style={
	dashed,
	very thick
    }
}

\scalebox{1}{
\begin{tikzpicture}
\newcommand{\rechts}{7}
\newcommand{\radius}{1.9cm}
\coordinate (centerTop) at (\rechts,1.2);
\coordinate (centerLinks) at (0,0);
\coordinate (centerRechts) at (\rechts,0);
\coordinate (centerBottom) at (\rechts,-3);

% Linker Kreis
%\draw (centerLinks) circle (1.5*\radius) node [text=black] 
%{On-Premise-Lösung};
\node[punkt] {On-Premise-Lösung} ; 

% Oberer Kreis
\draw[fill=green] (centerTop) circle (\radius);
\node[align=left] at ($ (centerTop) + (3.5,1)$) {Chancen};

%Unterer Kreis mit Schrift
\draw[fill=red] (centerBottom) circle (\radius);
\node[align=left] at ($ (centerBottom) + (3.5,0)$) {Risiken};


% Rechte Mitte
%\draw (centerRechts) circle (0.8*\radius);
\node[align=left] at ($ (centerRechts) + (3.7,0)$) {Cloud-Lösung};
\node [cloud, draw,cloud puffs=10,cloud puff arc=120, aspect=2, inner 
ysep=2em] (CLOUD) at (centerRechts) 
{};


% Pfeil
\path[draw=black,solid,line width=1mm,fill=black,
preaction={-triangle 90,thin,draw,shorten >=-1mm}
] ($ (centerLinks) + (1.6 * \radius,0) $) -- ($ (centerRechts) + 
(-1.4 * \radius,0) $);

%\draw \secondcircle node [text=black,left] {On-Premise-Software};
%\draw \thirdcircle node [text=black,right] {$C$};
\end{tikzpicture}
}
\caption{Funktionsumfang vor und nach der Migration nach dem in dieser Arbeit 
vorgestellten Vorgehensweise. 
Selbsterstellte Grafik.}
\label{fig:funktionsumfang_agil}
\end{center}
\end{figure}



\begin{comment}
\subsubsection{Technische Machbarkeit}


\subsubsection{Wirtschaftliche Machbarkeit}
"`In diesem Punkt unterscheidet sich Cloud-Computing von früheren Paradigmen
wie Outsourcing, welches nicht auf das Geschäftsmodell des Unternehmens wirken
will. Durch neue Anwendungsszenarien kann mit Cloud-Computing ein beachtlicher
Mehrwert geschaffen werden."' \pcite{}{154}{cloud_migration}

"`Cloud-Computing verändert die Preisgestaltung von IT-Services und kann als
eine Reaktion  auf das straffere Kostenmanagement im Nachgang der Finanzkrise
2008 verwendet werden. Unternehmenskunden wollen sich nicht mehr mit fixen
IT-Kosten binden. Sie wollen auch in der IT des Unternehmens die Capital
Expenditures (CAPEX) in Operational Expenditures (OPEX) umwandeln. Dies ist der
Hauptgrund, wieso das 'Pay-per-Use-Pricing' von Cloud-Services bei den
 Unternehmen so viel Interesse geweckt hat."' \pcite{}{156}{cloud_migration}


"`Mögliche betriebswirtschaftliche Vorteile: bessere Nutzung des
Ressourcen-Pools des Unternehmens, Verbesserung der Zeit zur Markteinführung
neuer Produkte und Dienstleistungen, Steigerung der Agilität und letztlich
Verbesserung der Erfahrungen der Kunden im Umgang mit dem Unternehmen,
welches auf Cloud-Computing setzt."'
\pcite{}{157}{cloud_migration}


"`Alle Empfehlungen sollten mit einem Business Case hinterlegt werden, der
die Höhe der Kostenreduktion und die Verbesserung des Servicelevels zeigt."'
\pcite{}{158}{cloud_migration}
\end{comment}


\subsection{Phase III: Anforderungsanalyse und -Planung}
Der Software Engineering Prozess bei der Migration zu einem 
Software-as-a-Service-Modell ändert sich. 
\citepara{changes_in_requirements_engineering}

Anforderungsermittlung oder Requirementsengineering hat das Ziel Anforderungen 
eines Softwaresystem oder eines Features zu erheben, zu strukturieren, 
priorisieren und zu koordinieren. 
\pcite{}{}{changes_in_requirements_engineering} Auch die Identifizierung von 
Indiskrepanzen zwischen Ergebnis und Ziel gehört dazu. 
\pcite{}{}{requirements_engineering_process_for_saas_cloud_env}

Die Migration in die Cloud macht ein Überdenken der Software Engineering 
Prozesse notwendig. \pcite{}{}{transitioning_to_saas}

Besseres Requirements Engineering kann Kosten für die Fehlbehebung um 20\% 
senken.

\begin{figure}[!h]
\begin{center}
\includegraphics[width=0.8\textwidth]{images/soa_architecture.png}
\caption{Service-oriented architecture. Aus 
\protect\citeflow{changes_in_requirements_engineering}}
\label{fig:soa_architecture}
\end{center}
\end{figure}

\begin{figure}[!h]
\begin{center}
\includegraphics[width=0.8\textwidth]{images/iterative_se_process.png}
\caption{Iterativer Software Engineering Prozess. Aus 
\protect\citeflow{changes_in_requirements_engineering}}
\label{fig:iterative_se_process}
\end{center}
\end{figure}

Aus \citepara{changes_in_requirements_engineering}
\begin{itemize}
	\item Vergleich zwischen zwei Modellen des RE.
	\item Während Updates früher kostenlos (kleinere Updates) oder größere 
Updates (kostenpflichtiger Neuerwerb) waren, mieten Kunden im Cloud Umfeld eine 
Software und erwarten Updates.
	\item Bei einer Migration zu einem Service-Oriented-Architecture 
Modell, werden Teile aus einer Software geschnitten und über eine Schnittstelle 
als Dienstleistung angeboten. Daher wird in der Regel versucht, die meisten 
Funktionalitäten zu übernehmen. Für mehr Informationen: 
\pcite{}{}{service-oriented_migration}
	\item Es gibt drei Nicht-Funktionale Anforderungen die aus SaaS 
entstehen: 
\begin{enumerate}
	\item Software muss in einer Cloud-Umgebung gehostet werden. Dies 
hat Einfluss auf die Themen Sicherheit, Vertraulichkeit, Verschwiegenheit, 
Compliance.
	\item In der Regel eine webbasierte Anwendung, d.h. es werden übliche 
Internetprotokolle verwendet. Dies hat Einfluss auf die Themen: Multi-Tenancy, 
User Concurrency, Konfigurierbarkeit, Skalierbarkeit, Verlässlichkeit, 
Leistungsfähigkeit, Verfügbarkeit, Kompatibilität, Interoperabilität, 
Portabilität, Effizienz, Direktheit
	\item In der Regel ist eine Bedienung über den Browser möglich, die 
Installation einer zusätzlichen Anwendung nicht erforderlich
\end{enumerate}
	\item Business Changes erforderlich. Näheres 
\pcite{}{}{transitioning_to_saas}. Update Mechanismen, Non-disruptive Upgrade 
Modelle
	\item \pcite{}{}{requirements_engineering_process_for_saas_cloud_env} 
Neuer Stakeholder: Cloud Service Provider. Schlagen Checkliste für neue 
Stakeholder vor.
	\item Noch mehr Stakeholder Analysten, Grafik Designer, Kunden, 
Marketing, Sicherheitsexperten 
\pcite{}{}{adapting_the_software_engineering_process}
\end{itemize}

\subsubsection{Transformation des bisherigen Requirements-Engineering}
\citeflow{changes_in_requirements_engineering} schlagen die folgenden Schritte 
vor, um die Anforderungsermittlung an die Cloud anzupassen:
\begin{enumerate}
	\item Paradigma sich schnell ändernder Anforderungen etablieren. Die 
Nähe zur agilen Entwicklung und neuen Methoden der Anforderungsermittlung 
sorgen für eine Volatilität der Anforderungen.
	\item Integrieren der Anforderungsermittlung in einen iterativen, 
inkrementellen Software Engineering Prozess. Nicht typisch für die Cloud, aber 
besonders erforderlich aufgrund der volatilen Anforderungen.
	\item Identifizierung und Priorisierung der Stakeholder mit 
systematischen Methoden
	\item Einbeziehung der Kunden in die Anforderungsermittlung mittels 
Feature Requests und Bug Reports sind essentiell um Vorteil aus der 
Cloud-Migration zu ziehen. Nutzer müssen das Gefühl bekommen, Einfluss auf die 
Entwicklung nehmen zu können.
	\item Implementierung von Feedbackmöglichkeiten. Entweder automatisiert 
über die Auswertung von Daten oder über Formulare. 
	\item Nutzen von Mechanismen für unterbrechungsfreie Updates. Nicht 
Administratoren, sondern Entwickler bestimmen den Updatezeitpunkt.
	\item Entwicklung von Möglichkeiten, Software an kleineren 
Nutzergruppen zu testen.
\end{enumerate}


\subsubsection{Unterschiede im Requirements Engineering}
\begin{table}[ht!]
\centering
\begin{longtable}{|p{0.45\textwidth}|p{0.45\textwidth}|}
\hline
\textbf{On-Premise-Software} & \textbf{Software as a Service} \\
\hline %%%%%%%%%%%%%%%%%%%%%%%%%%%%%%%%%%%%%%%%%%%%%%%%%%%%%%%%%%%%%%%%%
Überschaubare Anzahl von Stakeholdern & Viele verschiedene Stakeholder \\
\hline
Keine oder geringe Einbeziehung des Kunden in die Entwicklung & Starke 
Einbeziehung des Kunden in die Entwicklung\\ \hline
Geschäftsbeziehung zum Kunden endet mit einmaliger Zahlung & langfristige 
Geschäftsbeziehung \\ \hline
Nutzungserfahrungen nur über spezielle Erhebungen & Direkte Rückmeldungen durch die Kunden, ggf.
motiviert durch die Hoffnung auf Fehlerbehebung oder neue Features \\ \hline
Regelmäßige, geplante Fehlerbehebung & Sofortige Fehlerbehebung \\ \hline
Keine neuen Features ohne Versionsupgrade & Andauernde Auslieferung neuer 
Features, ohne größere Verzögerung \\ \hline
Updates und Upgrades erfordern Downtime & Nahtloser Updateprozess ohne 
Unterbrechung \\ \hline
Upgrades haben größere Auswirkungen und machen Schulungen erforderlich & 
Kontinuierliche Auslieferung weniger disruptiv \\ \hline
Prognose und Test der Akzeptanz schwierig & Alternativen lassen sich an 
kleineren Nutzergruppen testen \\
\hline %%%%%%%%%%%%%%%%%%%%%%%%%%%%%%%%%%%%%%%%%%%%%%%%%%%%%%%%%%%%%%%%%
\end{longtable}
\caption{Unterschiede im Requirementsengineering. Entnommen aus 
\cite{changes_in_requirements_engineering}}
\label{tab:unterschiede_im_re}
\end{table}

\subsubsection{Kategorien und Anforderungen}
Aus \citeflow{requirements_engineering_process_for_saas_cloud_env}:
\begin{description}
	\item[Anforderungen an die Architektur] Passendes Auslieferungsmodell, 
Sicherheit, Privatsspähre, Hohe Verfügbarkeit und Anpassbarkeit. Verschiedene 
Levels von Service Level Agreements (SLA). Sollte zustandslos (niedrige Kosten, 
hohe Zuverlässigkeit und Performanz) sein. Fehlertolerant, umfassende 
Redundanz und Uptime, Strategien für den Fehlerfall. Wiederbenutzbare 
Komponenten. Integration mit Altsystemen, API Anforderungen. 
	\item[Anforderungen aus dem Operativen/an das Verhalten] Leichte 
Anpassbarkeit und Erweiterbarkeit der Interfaces, Daten- und Businessprozesse, 
User Experience, Designanforderungen, Sicherheit und Privatsspähre, 
Verfügabarkeit, Performanz, Interoperabilität, häufige und nicht unterbrechende 
Upgrades. Kapazitätsplanung, Datenmigration, 
	\item[Anforderungen des Managements] Zentralisierte Berichte, 
Monitoring über Einhaltung der SLAs, Rechnungen, Hosting, User Management, 
Kapazitätsplanung und Zuteilung, Datenmanagement, Speicherung, Verarbeitung
	\item[Anforderungen aus Technik/Implementierung] Recruitment, 
Rollenwechsel, 
\end{description}

Schritte, die laut 
\citeflow{requirements_engineering_process_for_saas_cloud_env} im klassischen 
Requirements Engineering ergänzt werden sollten:
\begin{description}
	\item[Testen der Eignung der Cloud-Architektur mit initialen 
Anforderungen]
	\item[Public oder Private Cloud?]
	\item[Passende Cloud Service Provider identifizieren]
	\item[Kosten der Cloud-Lösung abschätzen] Cloudonomics
	\item[Anforderungen dokumentieren] Wekcge Art von physischer und 
persönlicher Sicherheit bietet die CLoud? Wie würden Anwendungen überwacht, die 
in einer Public Cloud? Wie skalierbar wäre die Cloud?
\end{description}



\subsubsection{Anforderungsermittlung}
\subsubsection{Return on Investment (ROI)}
\subsubsection{Total Cost of Ownership (TCO)}
\begin{comment}
\subsection{Phase IV: Migration}
\subsection{Phase V: Tests und Auslieferung}
\subsection{Phase VI: Überwachung und Wartung}
\subsection{Gesamtbetrachtung}
\subsubsection{Agilität}



\begin{comment}
\label{cha:entwicklung}
Dieses Kapitel dient der Entwicklung eines konzeptuellen Rahmens auf Basis theoretischer Grundlagen, vorausgesetzt sie verfolgen einen positivistischen Ansatz. Hierfür leiten Sie Hypothesen aus verschiedenen sinnvoll kombinierten Quellen her. Hierdurch generieren Sie aus bestehendem Wissen neues Wissen, was eine Eigenleistung und somit ein wichtiger Bestandteil Ihrer Arbeit darstellt.

Sollte Ihre Arbeit nicht positivistisch ausgelegt sein, stellt dieser Abschnitt kein Pflichtkapitel der Arbeit dar. Alternativ beschreiben Sie Anforderungen für ein mögliches Konzept oder verzichten vollständig auf dieses Kapitel.

\textbf{Setzen Sie sich frühzeitig mit Ihrem Betreuer in Verbindung, um Ihre Gliederung abzustimmen und mögliche Missverständnisse zu beseitigen.}

Im Folgenden werden einige allgemeine Hinweise zu den Themen richtiges Zitieren und Literaturrecherche gegeben.


\subsection{Quellen und richtiges Zitieren}
Quellen können in Fußnote oder direkt im Text platziert werden. Alles was nicht Ihr eigenes Gedankengut darstellt, muss mit einer entsprechenden Quelle belegt werden. Hierbei können wörtliche und indirekte Zitate verwendet werden. Wörtliche Zitate sind immer mit der Seitennummer der Quelle anzugeben.

Beispiel für ein direktes Zitat:

%\textit{\glqq The case study is a research strategy which focuses in understanding the dynamics present within single settings\grqq} \pcite{}{543}{eisenhardt1989}.

Beispiel für ein indirektes Zitat:

%Eine explorative Fallstudie dient der Gewinnung von neuen Erkenntnissen und der Bildung von neuen Hypothesen über bestimmte Sachverhalte. Durch den Beitrag zum Theorieaufbau ist der Erkenntnisgewinn höher als bei einer reinen deskriptiven Fallstudie. In explorativen Fallstudien werden Phänomene in noch wenig erforschten Gebieten identifiziert und aus erkannten Zusammenhängen neue Hypothesen gebildet \citepara{eisenhardt1989}.

Alternativ kann die Quelle auch im  laufenden Text angegeben werden:

%Nach \citeflow{eisenhardt1989} wird die Wichtigkeit der Fallauswahl oft unterschätzt. Die Fälle können zwar zufällig ausgewählt werden, dies ist aber weder notwendig noch wünschenswert.

Quellenangaben bestehen aus Autor, Jahr und ggf. Seitenangabe. Bei zwei Autoren sind beide Autoren zu nennen, bei mehreren Autoren nur der erste Autor mit dem Zusatz „et al.“.\newpage


\subsection{Zitieren mit Endnoten}
Im Rahmen der Erstellung von Arbeiten am Fachgebiet ISE ist das Literaturverwaltungsprogramm EndNote zu verwenden. Dieses steht auf der \href{http://www.ulb.tu-darmstadt.de/service/literaturverwaltung_start/endnote_ulb/endnote.de.jsp
}{ULB-Seite} zum Download verfügbar.


\subsubsection{Lateinischer Text mit Zitaten für Erstellung des Literaturverzeichnisses}
\label{cha:source:latintext}
%Lorem ipsum dolor sit amet, consectetur adipiscing elit. Sed vitae lacus eu
%augue semper lobortis vitae aliquet leo. Fusce eleifend sodales commodo
%\citepara{eisenhardt1989}. Mauris arcu metus, bibendum sagittis condimentum
%eget, placerat a enim \citepara{baechle2010}. Quisque sit amet sagittis lectus.
%Curabitur sit amet libero eu felis elementum mollis. Nullam odio diam, mollis
%vitae viverra ut, laoreet ut odio. Praesent facilisis suscipit consequat. Morbi
%feugiat rutrum erat, eu sagittis nibh rhoncus nec \citepara{melao2000}.

%In euismod, arcu ut semper adipiscing, nibh odio ullamcorper arcu, ut scelerisque massa magna nec quam \citepara{benlian2013}. Curabitur bibendum nibh eget augue pellentesque iaculis \citepara{sheffi2005}. Praesent iaculis auctor gravida. Quisque congue, magna ut bibendum semper, enim tortor ultrices lorem, ac feugiat tortor lectus nec nunc \citepara{carnap1974}. Pellentesque habitant morbi tristique senectus et netus et malesuada fames ac turpis egestas. Lorem ipsum dolor sit amet, consectetur adipiscing elit. Fusce dignissim, augue a sodales tristique, neque dui mollis arcu, id interdum augue justo sed lacus \citepara{welchering2013}. Vestibulum ante ipsum primis in faucibus orci luctus et ultrices posuere cubilia Curae; Mauris euismod bibendum nulla, sed accumsan urna tempor sed. Etiam eget diam eros, sed aliquet dolor \citepara{broadbent1996}. Phasellus vitae quam in orci convallis pharetra. Donec sit amet imperdiet nisi \citepara{kayser2013}. Sed vel interdum orci. Praesent vulputate, dolor id varius egestas, enim libero cursus neque, a cursus sapien nulla ut augue. Nullam vitae tortor nisl, vitae cursus enim. Suspendisse eget metus ipsum, sit amet varius sem \citepara{shazly2013}.

\subsection{Literaturrecherche}
Anbei eine kurze Auflistung von möglichen Kanälen zur Literaturrecherche.

\textbf{Zu Verwaltung Ihrer Literatur benutzen Sie bitte das Programm EndNote, dieses wird kostenfrei von der TU zu Verfügung gestellt.}

\url{http://www.ulb.tu-darmstadt.de/angebot/service/literaturverwaltung/endnote.de.jsp}

\subsubsection{Angebot der ULB}
\begin{itemize}
\item Universitätsbibliotheken (\url{http://www.ulb.tu-darmstadt.de/})
\item Rechercheangebot der ULB (\url{http://www.ulb.tu-darmstadt.de/recherche/})
\end{itemize}

\subsubsection{Online-Datenbanken und -Bibliotheken}
\begin{itemize}
\item Elektronische Zeitschriftenbibliothek (EZB) \\
(\url{http://rzblx1.uni-regensburg.de/ezeit/fl.phtml?bibid=TUDA})
\item AIS Electronic Library (AISeL)\\
(\url{http://aisel.aisnet.org/})
\item Zeitschriftendatenbank (ZDB)\\
(\url{http://dispatch.opac.ddb.de/DB=1.1/srt=YOP/})
\item Datenbank-Infosystem (DBIS): Literatur- und Fakten-Datenbank\\
(\url{http://rzblx10.uni-regensburg.de/dbinfo/fachliste.php?bib_id=tud})
\item IEEE Xplore \\
(\url{http://ieeexplore.ieee.org/Xplore/dynhome.jsp?tag=1})
\item EBSCO: internationale wirtschafts-wiss. Zeitschriften\\ (\url{http://search.ebscohost.com})
\item Springer-Online: Bücher/Beiträge des Springer Verlags\\
(\url{http://www.springerlink.com})
\item WiSo Net: deutschsprachige Literatur zu Wirtschafts- und Sozialwissenschaften\\
(\url{www.wiso-net.de})

\end{itemize}

\subsubsection{Sonstiges}
\begin{itemize}
\item \textbf{Google Scholar:} Suchdienst für wissenschaftliche Recherchen (http://scholar.google.de)
\item \textbf{Verlagswebseiten} Recherche und den Zugriff auf Zeitschriften- und Zeitungsartikel und E-Books
\item \textbf{Webseiten von Unternehmen} für die Recherche von Unternehmensdaten und-statistiken sowie Unternehmensdatenbanken
\item \textbf{Webseiten von Bundes- und Landesbehörden sowie der EU}
 Statistisches Bundesamt (http://www.destatis.de)
\\Presse- und Informationsamt der Bundesregierung (http://www.bundesregierung.de)
\item \textbf{Webseiten von Marktforschungsinstituten}
(für Marktanteile und Verbraucheranalysen)
\item \textbf{Webseiten von Verbänden und Kammern}
Institut der deutschen Wirtschaft (http://www.deutsche-wirtschaft.de)
\end{itemize}
\end{comment}
