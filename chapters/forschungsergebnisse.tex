\section{Das Projekt: iFMS@Salesforce}
%\section{Forschungsergebnisse}
\label{cha:result}
\begin{comment}
In Kapitel „Forschungsergebnisse“ stellen Sie die Ergebnisse ihrer Arbeit dar.
An dieser Stelle nehmen Sie noch keine Interpretation oder Erläuterung der 
Ergebnisse vor, sondern beschreiben rein deskriptiv ihre Befunde. Eine 
Auswertung findet im nachfolgenden Kapitel statt.
\end{comment}

Mit dem entwickelten Vorgehensmodell soll nun beispielhaft ein Geschäftsmodell 
und eine Vision für die Migration des in Kapitel~\ref{cha:replyundifms} 
vorgestellten iFMS entstehen. Die Cloudversion dieser Software wird hier 
"`iFMS@Salesforce"' genannt und von iFMS abgegrenzt. Aus dem Namen geht die 
gewählte Cloud-Plattform hervor -- eine strategische Entscheidung, die im selben 
Kapitel begründet wurde.

\subsection{Realisierung von Chancen}
\begin{description}
	\item[Soziales Element: Vernetzung und Einbeziehung von Nutzern] Die 
Vernetzung zwischen den Nutzern einer Firma erfolgt über Chatter, dem 
Salesforce Äquivalent von Facebook. Der ISV bietet Kundensupport über die 
Salesforce Service Cloud. Entwickler sind für Kunden direkter erreichbar. Durch 
die dort gesammelten Erfahrung können zukünftige Entwicklungsarbeiten 
zielgerichteter erfolgen.
	\item[Analysemöglichkeiten] Der ISV hat Zugriff auf die 
Salesforceinstanzen seiner Kunden. Dadurch kann er Fehlkonfigurationen 
und Fehlbedienungen erkennen, Supportangebote verbessern oder 
Benutzeroberflächen intuitiver gestalten.
	\item[Mobile Nutzung] Mit der Mobile App Salesforce1 ist die Salesforce 
Anwendung auch auf Mobilgeräten nutzbar. Damit ein echter Mehrwert entsteht, 
sollen sich
\begin{itemize}
	\item Rauminformationen wie Belegungen und Ansprechpartner abrufen 
lassen.
	\item Räume buchen lassen.
	\item Schadensmeldungen eingeben lassen und nach Freigabe durch einen 
verantwortlichen Mitarbeiter in einem Auftrag an einen Dienstleister 
resultieren.
	\item über Anwesenheitserkennung Raumtemperatur und Licht regeln lassen.
	\item Räume über ein Indoornavigationssystem finden lassen.
	\item Reiningungspläne einsehen und ändern sowie Sonderreinigungen 
anfordern lassen.
\end{itemize}

	\item[Reduzierte Markteintrittskosten \& Skalierte Märkte] Um die 
Hürden für potentielle Kunden gering zu halten und das Produkt auf diese Weise 
auch für kleine und mittlere Unternehmen attraktiv zu machen, ist die 
SAP-Anbindung optional. Durch die Nutzung des Salesforce Marktplatzes für Apps 
können Kunden das Produkt leicht beziehen und testen.
	\item[Skalierung der Leistung] Raumbelegungen lassen sich über die 
verfügbare Leistung in akzeptabler Zeit optimieren. CAD-Pläne werden nicht 
mehr lokal sondern in der Cloud konvertiert.
	\item[Time to market \& kürzere Releasezyklen] Ein vermarktbares 
Minimum mit einem Bruchteil der Funktionalitäten der komplexen Altsoftware soll 
rasch entwickelt und vermarktet werden. Indem weitere Entwicklung auf den 
Markterfahrungen beruht, wird schnell viel Wert für Kunden geschaffen.
	\item[Alternativen am Kunden testen] 
	\item[Wartung einer einzigen Version] Es gibt einen einzigen 
Softwarestand auf Salesforce der in besonderem Service für Kunden angepasst 
wird. 
	\item[Standardisierte Komponenten] Standardisierte Komponenten lassen 
sich auf dem Salesforce App Marktplatz beziehen. Beispiel dafür ist die 
nahtlose Anbindung an Amazons Dateispeicher S3. Zu prüfen ist auch, ob sich 
Import, Anzeige, Veränderung, Export von CAD Dateien über eine eingekaufte 
Komponente realisieren lassen, da das Unternehmen in diesem Bereich keine 
Kernkompetenz besitzt.
	\item[Stetige Umsätze] Durch ein Preismodell in dem pro Nutzer und 
Monat abgerechnet wird, lassen sich stetige Umsätze generieren.
	\item[Verkauf an Fachabteilungen] Im Marketing soll ein Schwerpunkt 
darauf gelegt werden, Fachabteilungen direkt zu erreichen, zum Beispiel bei 
Fachtagungen einschlägiger Themen. Idealerweise lassen sich kleine Firmen 
erreichen, die das Thema Liegenschaftsmanagement bisher ohne 
Softwareunterstützung bewältigt haben. Sind Fachabteilungen überzeugt lässt 
sich das Produkt ohne Eingriffe in die IT-Landschaft vor Ort nutzen.
\end{description}

\subsection{Vermeidung von Risiken}
\begin{description}
	\item[Hohe Kosten durch Pay-per-Use] Salesforce rechnet pro Nutzer und 
Monat ab, sodass weder für Arlanis Reply noch für den Kunden mit unerwartet 
hohen Kosten zu rechnen ist.
	\item[Lock-in Effekte] Arlanis Reply hat sich strategisch an Salesforce 
gebunden; der Lock-in Effekt wird in Kauf genommen.
	\item[Komplexität unbedacht gekoppelter Komponenten]
	\item[Datenmigration]
	\item[Leistungstransparenz]
	\item[Geringere Umsätze] Stetige, aber geringere Umsätze werden durch 
einmalige Zahlungen für Anpassungen ergänzt. 
	\item[Geringere Anpassbarkeit] Mit Heroku wird Salesforce um PaaS 
ergänzt; Features, die sich nicht mit der Kernfunktionalität oder Erweiterungen 
umsetzen lassen, lassen sich auf diese Weise nahezu ohne Einschränkungen 
realisieren.
	\item[Organisatorische und strukturelle Umbrüche] Da
	\item[Updatefrequenz erfordert Agilität] 
\end{description}

\subsection{Geschäftsmodell \& Strategie}
\begin{comment}
In Orientierung an Abbildung~\ref{fig:einfluss_des_preises_auf_business} soll 
in Tabelle~\ref{} zunächst iFMS in Preis und davon abhängigen Dimensionen 
analysiert werden. 


\begin{table}[bh]
\newcommand{\colBreite}{0.135\textwidth}
\newcommand{\theColor}{blue!25}
\centering
\begin{tabular}[width=0.95\textwidth]
{|p{\colBreite}|p{\colBreite}|p{\colBreite}|p{\colBreite}|p{
\colBreite} |p{\colBreite}|}
\hline
\textbf{Verkaufs\-preis} & \textbf{Kosten für Kun\-den\-aqui\-si\-tion} & 
\textbf{Vertrieb} & \textbf{Kun\-den\-be\-ziehungen} & \textbf{Vertretbare 
Nutzerschulungen} & \textbf{Ziel\-gruppen} \\
\hline
€€€€€ & Hoch & Extern & \cellcolor{\theColor}Persönlich\newline iFMS & 
\cellcolor{\theColor}Erheblich\newline iFMS & 
\cellcolor{\theColor}Unterneh\-men\newline iFMS \\
\hline
\cellcolor{\theColor}€€€\newline iFMS & \cellcolor{\theColor}Moderat\newline 
iFMS & \cellcolor{\theColor}Intern\newline iFMS & Telefonisch & Moderat & 
Abteilungen, 
KMU \\
\hline
€ & Niedrig & Keinen & Web & Minimal & KMU, Personen \\
\hline
\end{tabular}
\caption{Kosten für iFMS und iFMS@Salesforce aus Kundensicht und Folgerungen 
auf die Zielgruppe.}
\label{tab:analyse_preis}
\end{table}
\end{comment}
Die Bedeutung für das Geschäftsmodell und die Strategie soll beispielhaft an 
einem Aspekt gezeigt werden, der Zielgruppe. Die Zielgruppe von iFMS bestand 
aus großen Unternehmen mit einer bestimmten 
Zahl von Liegenschaften, für deren Verwaltung sich Einrichtungs-, Wartungs- und 
Schulungsaufwände lohnen. Die Beschränkung auf diese Zielgruppe ließe sich 
falls gewünscht mit den Mitteln der Cloud aufheben. Eine Basisversion könnte 
durch eine einfache Installation und begeisternde Cloud-Features zu einer 
relativ niedrigen monatlichen Gebühr in einer Art Freemium-Modell Neukunden 
erreichen. Die Unterstützung für SAP und CAD, spezielle Anpassungen der 
Software, Integration weiterer Systeme, Support sowie Schulungen ließen sich 
durch Kunden hinzubuchen und machten die Hauptumsatzquelle für das Unternehmen 
aus. Dabei wäre darauf zu achten, den persönlichen Kontakt zu Kunden nicht zu 
verlieren: Das Verarbeiten von CAD-Plänen mit sicherheitsrelevanten Daten oder 
Finanzdaten aus SAP mit personenbezogenen Daten in der Cloud und macht es 
erforderlich, dass der Kunde Salesforce und dem ISV vertraut.

\begin{comment}
Um mögliche Strategien, Geschäftsmodelle und 
Zielgruppen für iFMS@Salesforce aufzuzeigen, sollen diese Aufwände in Anlehnung 
an Abbildung~\ref{fig:einfluss_des_preises_auf_business} zunächst analysiert 
werden; die Zusammenfassung der Ergebnisse ist in 
Tabelle~\ref{tab:kosten_fuer_ifms} zu finden.
 \begin{table}[bh]
\centering
\begin{tabular}[width=0.9\textwidth]
{|p{0.15\textwidth}|p{0.37\textwidth}|p{0.37\textwidth}|}
\hline
\textbf{Aufwand} & \centering\textbf{iFMS} & 
{\centering\textbf{Möglichkeiten\newline für iFMS@\-Salesforce}} \\
\hline
Infrastruktur & Bereitstellung und Wartung der Soft\-ware auf 
Servern und Clients durch den Kunden & Für Kunden reduziert auf monatliche 
Zahlungen pro Nutzer an Salesforce\\
\hline
Preis und Lizenzpolitik & Kaufpreis und Wartungsverträge & Lizenzkosten \\
\hline
Anpassungen an Software & Kerngeschäft & Als Kerngeschäft oder Nebengeschäft? \\
\hline
Schulungen & Kerngeschäft & Abhängig vom Preismodell \\
\hline
Kosten bei Nutz\-ung & Hoher Aufwand für Pflege der CAD-Pläne und Daten in 
Verbindung mit SAP-System & Lässt sich Liegenschaftsmanagement ohne CAD 
und/oder SAP verkaufen? \\
\hline
Zielgruppe & $\Rightarrow$ Große Unternehmen & $\Rightarrow$ unter Umständen 
auch KMU \\
\hline
\end{tabular}
\caption{iFMS, seine Aufwände und Möglichkeiten für iFMS@Salesforce}
\label{tab:kosten_fuer_ifms}
\end{table}
% $\surd$

\begin{comment}
\subsubsection{Sonstiges}
\begin{itemize}
\item \textbf{Google Scholar:} Suchdienst für wissenschaftliche Recherchen 
(http://scholar.google.de)
\item \textbf{Verlagswebseiten} Recherche und den Zugriff auf Zeitschriften- 
und 
Zeitungsartikel und E-Books
\item \textbf{Webseiten von Unternehmen} für die Recherche von 
Unternehmensdaten 
und-statistiken sowie Unternehmensdatenbanken
\item \textbf{Webseiten von Bundes- und Landesbehörden sowie der EU}
 Statistisches Bundesamt (http://www.destatis.de)
\\Presse- und Informationsamt der Bundesregierung 
(http://www.bundesregierung.de)
\item \textbf{Webseiten von Marktforschungsinstituten}
(für Marktanteile und Verbraucheranalysen)
\item \textbf{Webseiten von Verbänden und Kammern}
Institut der deutschen Wirtschaft (http://www.deutsche-wirtschaft.de)
\end{itemize}
\end{comment}

Die Einstiegshürden

Die infrastrukturellen Aufwände werden sich reduzieren. Auf Seite des Kunden, 
weil die Software nicht auf eigenen Servern sondern bei Salesforce läuft und 
für die Nutzung ein Pauschalbetrag pro Nutzer und Monat fällig ist. Auf Seite 
des ISV, weil es nur noch eine zu unterstützende Plattform gibt.


Auch wenn sich mit einer Realisierung auf 
Salesforce Einrichtungs- und Wartungskosten für die Software auf 
Kundenseite deutlich reduzieren lassen, bleiben Aufwände für CAD-Plan- und 
Datenpflege. 
\end{comment}