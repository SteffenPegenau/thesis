\section{Diskussion des modellierten Vorgehens aus praktischer Sicht}
\label{cha:diskussion}
\begin{comment}
Im vorletzten Abschnitt diskutieren Sie Ihre Ergebnisse und stellen den Beitrag 
für die Praxis und für die Forschung dar. Gehen Sie auch auf die Einschränkungen 
Ihrer Arbeit ein.
\end{comment}
Im folgenden werden die in den durchgeführten Einzelgesprächen identifizierten 
Unterschiede zwischen dem tatsächlichen und dem aus dem Modell heraus 
vorgeschlagenen Vorgehen vorgestellt und aufgezeigt, an welchen Stellen und in 
welche Richtungen das Modell erweitert werden müsste.

\subsection{Vermarktung als Projekt}
Der wesentlichste Unterschied zwischen dem vorgeschlagenen und 
tatsächlichen Vorgehen besteht in der Vermarktung. Das mit dem Modell 
erarbeitete Vorgehen sieht vor, Kunden dauerhaft zu binden und dauerhafte 
Umsätze zu generieren. Diese dauerhafte Bindung will Arlanis Reply aus 
strategischen Gründen mittelfristig abbauen, da mit ihr ebenso dauerhafte 
Verpflichtungen eingehen: Kunden erwarten sich von ihren regelmäßigen Zahlungen 
auch stetige Verbesserungen, die aus Sicht des Managements einen 
unverhältnismäßig hohen Aufwand bedeuten. Deshalb will Arlanis iFMS@Salesforce 
in Form einzelner Projekte verkaufen. Anstatt eine fertige Basisversion und 
gegebenenfalls CAD-/SAP-Erweiterungen im Salesforce Marktplatz zu kaufen, 
erhält der Kunde gegen eine einmalige Zahlung eine Software, die von 
Arlanis-Consultants und -Entwicklern in dessen Salesforceumgebung eingerichtet 
wird. 
Jede Erweiterung und Verbesserung muss vom Kunden beauftragt und bezahlt werden.
\begin{table}[h]
\centering
\begin{tabular}{|p{0.47\textwidth}|p{0.47\textwidth}|}
\hline
\textbf{Vorgeschlagenes Vorgehen:} & \textbf{Tatsächliches Vorgehen} \\
\hline
Vermarktung eines weitgehend standardisierten Produktes & Vermarktung eines für 
einen Kunden angepassten Produktes \\
\hline
Bezug über Salesforce-Marktplatz & Produkt wird für den Kunden in 
seiner Installation eingerichtet \\
\hline
Drei Komponenten: Basis, CAD und SAP & Monolithische Architektur, die den 
Wünschen des Kunden entspricht \\
\hline
Kunden erhalten über Marktplatz immer die aktuellste Version & Kunden erhalten 
nur Anpassungen und Updates, wenn sie dafür zahlen \\
\hline
Entwicklung, ohne konkreten Kunden für das Produkt zu haben & Entwicklung 
eines jeden Features ausschließlich, wenn es zahlenden Kunden gibt, der 
Entwicklung in Auftrag gibt \\
\hline
\end{tabular}
\caption{Unterschiede zwischen tatsächlichem und vorgeschlagenem Vorgehen}
\label{tab:unterschiede_im_vorgehen}
\end{table}
Aufgrund des im Vergleich zum modellierten Vorgehen geänderten 
Geschäftsmodells, ändert sich das gesamte Produkt (Vgl. 
Tabelle~\ref{tab:unterschiede_im_vorgehen}): Es muss keine allgemein 
funktionierende Basisversion geschaffen werden, die Kunden dazu animiert, 
Erweiterungen zu kaufen, da eine monolithische, auf die jeweiligen 
Kundenbedürfnisse abgestimmte Architektur gewählt wird.

Im Sinne dieser Ausrichtung begann die tatsächliche Entwicklung nicht in der 
Hoffnung das Produkt einmal vermarkten zu können, sondern mit einem konkreten 
Kunden (K), der bereit war für die Entwicklung einer auf ihn angepassten 
Version von iFMS@Salesforce zu zahlen. Bei der künftigen Entwicklung 
beziehungsweise Anpassung von iFMS@Salesforce für andere Kunden soll auf diese 
für K entwickelte Basis zurückgegriffen werden. Für andere Kunden getätigte 
Entwicklungen fließen aber nicht automatisch zurück an K, sondern werden diesem 
in einem Angebot zum Kauf vorgeschlagen.

Aus dieser Einzelvermarktung folgt, dass die gesammelten Ideen dem Kunden zwar 
vorgeschlagen werden können, die Implementierung aber von dessen 
Zahlungsbereitschaft abhängt.

Diese Erfahrung bestätigt das enge Wechselspiel zwischen dem Geschäftsmodell, 
dem geplanten Leistungsumfang sowie der Chancen und Risiken.

\subsection{Skaleneffekte}
Während der Modellvorschlag auf die Erzielung von Skalenerträgen durch die 
Vermarktung auf dem Salesforce Marktplatz setzt, sind aufgrund der Vermarktung 
als Projekt Verkauf, Anpassung und Training der Endnutzer weiterhin sehr 
personalintensiv, die Skalierbarkeit des Geschäftsmodells dadurch 
eingeschränkt. 

Durch die Vermarktung als Projekt steigt für den Kunden die Investition und mit 
ihr das Risiko. In Folge werden -- anstatt das Produkt direkt an die 
Fachabteilungen zu vermarkten -- weiterhin in vielen Gremien langwierige 
Verhandlungen um den Leistungsumfang geführt, die in Lasten- und Pflichtenheften 
resultieren. Mit diesem Verlust an Agilität geht auch ein weiterer Verlust an 
Skalierbarkeit einher. Anstatt einem Kunden schnell die Lösung zu liefern und 
Umsätze zu generieren, muss der ISV erheblich in die Verhandlungen investieren, 
ohne das der Vertragsabschluss gesichert wäre.

Während eine personalintensive Wertschöpfung charakteristisch für 
Beratungsunternehmen ist, könnte Arlanis überdenken, ob der direkte Verkauf an 
Fachabteilungen durch Minderung des Risikos für den Kunden durch die 
Realisierung von Cloud-Vorteilen forciert werden könnte. 

\subsection{Anforderungsermittlung und -umfang}
Im Zuge der Anforderungsermittlung für iFMS@Salesforce traten 
unterschiedlichen Ansichten zwischen den Entwicklern (des alten iFMS) und 
dem Management zu Tage. Während den Entwicklern eine möglichst elegante 
Architektur am Herzen lag, die es in Zukunft ermöglichen würde, die gleiche 
Basis für eine Vielzahl von Kunden nutzen zu können, war dem Management vor 
allem eine schnelle und den Kunden K zufriedenstellende Lösung (Minimum viable 
product) wichtig. Eine allgemeine Lösung würde -- da K dafür nicht zahlt -- 
zunächst auf Kosten für Arlanis Reply hinauslaufen. Man war nicht bereit für 
eventuelle zukünftige Kunden in Vorleistung zu treten. Eine Entscheidung die vom 
Modell dieser Arbeit gestützt wird, um das Risiko für den ISV zu minimieren.

Ein ähnlich gelagertes Problem entstand aus der sentimentalen Bindung der 
Entwickler zu ihrer bisherigen Arbeit. Mit dem Übergang zu Salesforce wird der 
größte Teil ihrer bisherigen Arbeit nicht mehr benötigt. Aufgrund des im 
Vergleich zum alten iFMS reduzierten Leistungsumfangs wird zudem die neue 
Software viel weniger Fähigkeiten haben. Es zu erreichen, dass die Entwickler 
trotzdem überzeugt hinter dem neuen Produkt stehen, stellt eine Herausforderung 
dar, die in dem hier vorgestellten Modell nicht berücksichtigt wird.

\subsection{Forschungsbeitrag}
Diese Arbeit konnte mit Ihren Ergebnissen zeigen
\begin{enumerate}
	\item welche Merkmale, Chancen und Risiken bei der Migration in die 
Cloud zu berücksichtigen sind,
	\item wie diese Merkmale die strategische Ausrichtung und das 
Geschäftsmodell von Softwareherstellern beeinflussen und
	\item wie diese Merkmale den Migrationsprozess von Softwareherstellern 
beeinflusst
\end{enumerate}
und damit die in Kapitel~\ref{cha:einleitung} gestellten Forschungsfragen 
beantworten, sowie wechselseitige Einflüsse zwischen Geschäftsmodell, Strategie 
und Merkmalen aufzeigen.

Der Katalog der Cloud-Merkmale half bei einem Projekt aus der Praxis als 
Checkliste zu bedenkender Aspekte dabei, das geplante Vorgehen systematisch zu 
überprüfen. Dass die Ergebnisse zwischen vorgeschlagenem und tatsächlichen 
Vorgehen deutlich voneinander abweichen, ist unerheblich. Ziel dieser Arbeit war 
es, ein Vorgehensmodell zu entwickeln, das die Migration in die Cloud 
unterstützt. Dies geschieht indem Fragen aufgeworfen und Herausforderungen 
aufgezeigt werden, die Unternehmen für ihre Projekte individuell beantworten 
müssen. 

Auch wenn in dieser Arbeit Salesforce.com als Cloud-Anbieter aufgrund des 
Projektes aus der Praxis im Vordergrund stand, beschränkt sich das Projekt 
keineswegs auf diesen Anbieter. Tatsächlich sind einige Chancen der Cloud bei 
Salesforce technisch ausgeschlossen oder eingeschränkt. Zum Beispiel lassen 
sich bei Salesforce nicht automatisch verschiedene Implementierungsalternativen 
an Kunden testen und auswerten. 
Womöglich lässt sich auch die Entscheidung für 
einen Cloud-Anbieter fundierter treffen, indem verglichen wird, wie gut die 
einzelnen Anbieter die Realisierung der Chancen und die Vermeidung der Risiken 
unterstützen. Dies bleibt jedoch in einer anderen Arbeit zu prüfen.

