\section{Diskussion des modellierten Vorgehens aus praktischer Sicht}
\label{cha:diskussion}
\begin{comment}
Im vorletzten Abschnitt diskutieren Sie Ihre Ergebnisse und stellen den Beitrag 
für die Praxis und für die Forschung dar. Gehen Sie auch auf die Einschränkungen 
Ihrer Arbeit ein.
\end{comment}
Die Kompetenzfelder der vier 
Gesprächspartner, die in Kapitel~\ref{cha:praxis} vorgestellt wurden, den 
Rahmen der Diskussion des Modells und des vorgeschlagenen Vorgehens 
bilden. Dies bedeutet allerdings nicht, das zu einer Betrachtungsweise nur der 
jeweilige Gesprächspartner zur Sprache kommt.
\subsection{Übernahme der alten Funktionlitäten, insbesondere CAD \& SAP}
\subsection{Kundenkreis und Marketing}
\subsection{Möglichkeiten mit Salesforce}
\subsection{Strategische Ausrichtung}
Einer des wesentlichen Unterschiede zwischen dem vorgeschlagenen und 
tatsächlichen Vorgehen, besteht in der Vermarktung. Während das mit dem Modell 
erarbeitete Vorgehen darauf abzielt, Kunden langfristig zu binden, ist es bei 
Arlanis stragetisches Ziel diese Bindungen mittelfristig abzubauen. Es soll 
kein weitgehendes Standardprodukt über monatliche Zahlungen lizensiert werden, 

\begin{tabular}{|p{0.45\textwidth}|p{0.45\textwidth}|}
\hline
\textbf{Vorgeschlagenes Vorgehen:} & \textbf{Tatsächliches Vorgehen} \\
\hline
Vermarktung eines weitgehend standardisierten Produktes & Vermarktung eines für 
einen Kunden angepassten Produktes \\
\hline
Bezug über Salesforce-Marktplatz & Produkt wird für den Kunden in 
seiner Installation eingerichtet \\
\hline
Drei Komponenten: Basis, CAD und SAP & Monolithische Architektur, die den 
Wünschen des Kunden entspricht \\
\hline
Kunden erhalten über Marktplatz immer die aktuellste Version & Kunden erhalten 
nur Anpassungen und Updates, wenn sie dafür zahlen \\
\hline
Entwicklung, ohne konkreten Kunden für das Produkt zu haben & Entwicklung 
eines jeden Features ausschließlich, wenn es zahlenden Kunden gibt, der 
Entwicklung in Auftrag gibt \\
\hline
\end{tabular}
