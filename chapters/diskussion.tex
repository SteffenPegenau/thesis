\section{Diskussion des modellierten Vorgehens aus praktischer Sicht}
\label{cha:diskussion}
\begin{comment}
Im vorletzten Abschnitt diskutieren Sie Ihre Ergebnisse und stellen den Beitrag 
für die Praxis und für die Forschung dar. Gehen Sie auch auf die Einschränkungen 
Ihrer Arbeit ein.
\end{comment}
In diesem Kapitel soll der Beitrag des entwickelten Modell für die Praxis und 
die Forschung dargestellt werden. Dazu sollen zunächst exemplarisch drei 
Aspekte der tatsächlichen Umsetzung durch Arlanis Reply betrachtet werden: das 
Geschäftsmodell, weil Arlanis darin grundsätzlich vom vorgeschlagenen Vorgehen 
abwich, die Skalierbarkeit, weil dabei die Auswirkungen eines geänderten 
Geschäftsmodells deutlich werden und die Ermittlung des Leistungsumfangs, weil 
sie einen blinden Fleck dieses Modells aufzeigt.

\subsection{Geschäftsmodell: Projekt statt Produkt}
Arlanis Reply hat sich für eine Vermarktung von iFMS@Salesforce in Form von 
Projekten anstatt in Form eines Produktes entschieden und folgt damit seiner 
grundlegenden Strategie: Vermarktung von Anpassungen und Erweiterungen des 
Standardumfangs von Salesforce in Form von einzelnen Projekten. Dabei zahlt der 
Auftraggeber für einen vertraglich bestimmten Leistungsumfang. Im Gegensatz 
dazu müsste Arlanis bei der Entwicklung einer Software ohne konkreten Kunden 
zunächst investieren. Diese Investition ist mit Risiken verbunden, bietet aber 
auch die Chancen auf eine beliebige Skalierbarkeit: Die fertige Software lässt 
sich auf Online-Marktplätzen mit variablen Kosten nahe Null weltweit vertreiben.

Mit der Eingliederung des iFMS-Teams bekam Arlanis ein solches Produkt ins 
Portfolio, das jedoch nur im deutschsprachigen Raum bei persönlichen 
Verhandlungen vermarktet wurde. Die persönliche Betreuung setzte sich im 
Support fort: Für andauernde Umsätze durch Lizenzgebühren müsste Arlanis 
Mitarbeiter bereitstellen, die Support leisten, spontan auf kritische Fehler 
reagieren und das Produkt kontinuierlich weiterentwickeln können. Die 
Geschäftsführung befand daher, dass Arlanis zu wenige Mitarbeiter hat, um auf 
beiden Märkten -- Projekt- und Produktmarkt -- erfolgreich zu sein.

Aus diesem Grund sollte iFMS auf eine zweite Weise migriert werden: vom Produkt 
hin zum Projekt. Eine vollständige Migration von iFMS zu Salesforce war zu 
teuer und -- ohne gesicherte Kundschaft -- zu riskant. Zu dieser 
Risikoeinschätzung trug auch bei, dass sich die Geschaftsleitung unsicher war, 
ob sich iFMS@Salesforce verkaufen würde: CAD-Pläne können sicherheitskritische 
Informationen enthalten, die Kunden womöglich nicht in der Cloud wissen wollen. 
Ähnliches könnte für die SAP-Anbindung gelten. Unklar war auch, in 
welchem Umfang sich der Leistungsumfang des bisherigen iFMS auf Salesforce 
würde nachbauen lassen. Altkunden könnten auf einen Umstieg verzichten, wenn 
sie danach auf viele bekannte und gewohnte Funktionen verzichten müssen. 

Um das Risiko eines zu geringen Absatzes zu vermeiden, wurde ein Kunde K für 
ein Projekt iFMS@Salesforce gefunden. Anstatt eine fertige Basisversion und 
gegebenenfalls CAD-/SAP-Erweiterungen im Salesforce Marktplatz zu kaufen, 
erhält der Kunde gegen eine einmalige Zahlung eine Software, die von 
Arlanis-Consultants und -Entwicklern in dessen Salesforceumgebung eingerichtet 
wird. 
Jede Erweiterung und Verbesserung muss vom Kunden beauftragt und bezahlt werden.
\begin{table}[h]
\centering
\begin{tabular}{|p{0.47\textwidth}|p{0.47\textwidth}|}
\hline
\textbf{Vorgeschlagenes Vorgehen:} & \textbf{Tatsächliches Vorgehen} \\
\hline
Vermarktung eines weitgehend standardisierten Produktes & Vermarktung eines für 
den Kunden angepassten Produktes \\
\hline
Bezug über Salesforce-Marktplatz & Produkt wird für den Kunden in 
seiner Installation eingerichtet \\
\hline
Drei Komponenten: Basis, CAD und SAP & Monolithische Architektur, die den 
Wünschen des Kunden entspricht \\
\hline
Kunden erhalten über Marktplatz immer die aktuellste Version & Kunden erhalten 
nur Anpassungen und Updates für iFMS@Salesforce, wenn sie dafür zahlen (von Updates
an der Salesforce-Plattform abgesehen) \\
\hline
Entwicklung, ohne konkreten Kunden für das Produkt zu haben & Entwicklung 
eines jeden Features ausschließlich, wenn es zahlenden Kunden gibt, der 
Entwicklung in Auftrag gibt \\
\hline
\end{tabular}
\caption{Unterschiede zwischen tatsächlichem und vorgeschlagenem Vorgehen}
\label{tab:unterschiede_im_vorgehen}
\end{table}
Aufgrund des im Vergleich zum modellierten Vorgehen geänderten 
Geschäftsmodells, ändert sich das gesamte Produkt (Vgl. 
Tabelle~\ref{tab:unterschiede_im_vorgehen}): Es muss keine allgemein 
funktionierende Basisversion geschaffen werden, die Kunden dazu animiert, 
Erweiterungen zu kaufen, da eine monolithische, auf die jeweiligen 
Kundenbedürfnisse abgestimmte Architektur gewählt wird.

Sollten sich in Zukunft weitere Kunden für iFMS@Salesforce finden lassen, so 
wird für deren Implementierung auf die für K entwickelte Basis zurück 
gegriffen. Sollten Wünsche der neuen Kunden über die von K hinausgehen, fließen 
diese Verbesserungen und Erweiterungen jedoch nicht automatisch an K zurück. 
Aus dieser Einzelvermarktung folgt, dass die gesammelten Ideen dem Kunden zwar 
vorgeschlagen werden können, die Implementierung aber von dessen 
Zahlungsbereitschaft abhängt.

\subsection{Skalierbarkeit}
Während der Modellvorschlag auf die Erzielung von Skalenerträgen durch die 
Vermarktung auf dem Salesforce Marktplatz setzt, wäre dies bei der von Arlanis 
gewählten Strategie womöglich kontraproduktiv: Das Projektmodell rentiert sich 
aufgrund der höheren Preise, die man für individuelle Beratung, Implementierung 
und Anpassung verlangen kann -- höhere Preise, die Kunden eventuell nicht mehr 
bereit zu zahlen sind, wenn sie an der auf dem Salesforce Marktplatz 
bereitgestellten Variante feststellen, dass sie ein Standardprodukt erwerben.

Die Vermarktung in Form von einzelnen Projekten reduziert zunächst das Risiko, 
aber auch die Möglichkeiten dauerhaften, gleichmäßigen Umsatz zu generieren. 
Zudem erfordert die Neukundengewinnung erhebliche personelle Ressourcen bis ins 
Management hinein. Der persönliche Kontakt bei Beratung und Vertragsabschluss 
passt zum zu erzielenden hohen Preis (Vgl. 
Abbildung~\ref{fig:einfluss_des_preises_auf_business}); der Kunde erwartet den 
Kontakt zum Management. Die Einbeziehung größerer Personenkreise zieht die 
Vertragsverhandlungen in die Länge. Dadurch ist die Skalierbarkeit begrenzt. 
Sollte Arlanis nach erfolgtem Projekt mit K feststellen, dass ein Markt für ein 
Produkt iFMS@Salesforce existiert, kann auf Basis der für K entwickelten 
Umsetzung eine Version für den Marktplatz implementiert werden. Aufgrund des 
niedrigeren Preises und der kürzeren Bindungsdauer ist das Risiko für 
potentielle Kunden von Arlanis geringer. Entscheidungen für iFMS@Salesforce 
können von der nutzenden Fachabteilung getroffen werden, sodass der personelle 
Einsatz reduziert und die Implementierung besser skaliert werden kann. Der 
direktere Kontakt zu den eigentlichen Nutzern der Software kann zu mehr 
Agilität und Zufriedenheit führen.

\begin{comment}
Durch die Vermarktung als Projekt steigt für den Kunden die Investition und mit 
ihr das Risiko. In Folge werden -- anstatt das Produkt direkt an die 
Fachabteilungen zu vermarkten -- weiterhin in vielen Gremien langwierige 
Verhandlungen um den Leistungsumfang geführt, die in Lasten- und Pflichtenheften 
resultieren. Mit diesem Verlust an Agilität geht auch ein weiterer Verlust an 
Skalierbarkeit einher. Anstatt einem Kunden schnell die Lösung zu liefern und 
Umsätze zu generieren, muss der ISV erheblich in die Verhandlungen investieren, 
ohne das der Vertragsabschluss gesichert wäre.

Während eine personalintensive Wertschöpfung charakteristisch für 
Beratungsunternehmen ist, könnte Arlanis überdenken, ob der direkte Verkauf an 
Fachabteilungen durch Minderung des Risikos für den Kunden durch die 
Realisierung von Cloud-Vorteilen forciert werden könnte. 
\end{comment}

\subsection{Leistungsumfang und Anforderungsermittlung}
Im Zuge der Anforderungsermittlung für iFMS@Salesforce traten 
unterschiedlichen Ansichten zwischen den Entwicklern (des alten iFMS) und 
dem Management zu Tage. Während den Entwicklern eine möglichst elegante 
Architektur am Herzen lag, die es in Zukunft ermöglichen würde, die gleiche 
Basis für eine Vielzahl von Kunden nutzen zu können, war dem Management vor 
allem eine schnelle und den Kunden K zufriedenstellende Lösung (Minimum viable 
product) wichtig. Eine allgemeine Lösung würde -- da K dafür nicht zahlt -- 
zunächst auf Kosten für Arlanis Reply hinauslaufen. Man war nicht bereit für 
eventuelle zukünftige Kunden in Vorleistung zu treten. Eine Entscheidung die vom 
Modell dieser Arbeit gestützt wird, um das Risiko für Arlanis zu minimieren.

Ein ähnlich gelagertes Problem entstand aus der sentimentalen Bindung der 
Entwickler zu ihrer bisherigen Arbeit. Mit dem Übergang zu Salesforce wird der 
größte Teil ihrer bisherigen Arbeit nicht mehr benötigt. Aufgrund des im 
Vergleich zum alten iFMS reduzierten Leistungsumfangs wird zudem die neue 
Software viel weniger Fähigkeiten haben. Zu erreichen, dass die 
Entwickler trotzdem überzeugt hinter dem neuen Produkt stehen, stellt eine 
Herausforderung dar, die in dem hier vorgestellten Modell nicht berücksichtigt 
wird.

\subsection{Forschungsbeitrag}
Diese realen Überlegungen bestätigen das enge Wechselspiel zwischen dem 
Geschäftsmodell, dem geplanten Leistungsumfang sowie den Chancen und Risiken.
Diese Arbeit konnte mit Ihren Ergebnissen zeigen
\begin{enumerate}
	\item welche Merkmale, Chancen und Risiken bei der Migration in die 
Cloud zu berücksichtigen sind,
	\item wie diese Merkmale die strategische Ausrichtung und das 
Geschäftsmodell von Softwareherstellern beeinflussen und
	\item wie diese Merkmale den Migrationsprozess von Softwareherstellern 
beeinflusst
\end{enumerate}
und damit die in Kapitel~\ref{cha:einleitung} gestellten Forschungsfragen 
beantworten.

Der Katalog der Cloud-Merkmale half bei einem Projekt aus der Praxis als 
Checkliste zu bedenkender Aspekte dabei, das geplante Vorgehen systematisch zu 
überprüfen. Dass die Ergebnisse zwischen vorgeschlagenem und tatsächlichen 
Vorgehen deutlich voneinander abweichen, ist unerheblich. Ziel dieser Arbeit war 
es, ein Vorgehensmodell zu entwickeln, das die Migration in die Cloud 
unterstützt. Dies geschieht indem Fragen aufgeworfen und Herausforderungen 
aufgezeigt werden, die Unternehmen für ihre Projekte individuell beantworten 
müssen. 

Auch wenn in dieser Arbeit Salesforce.com als Cloud-Anbieter aufgrund des 
Projektes aus der Praxis im Vordergrund stand, beschränkt sich das Projekt 
keineswegs auf diesen Anbieter. Tatsächlich sind einige Chancen der Cloud bei 
Salesforce technisch ausgeschlossen oder eingeschränkt. Zum Beispiel lassen 
sich bei Salesforce nicht automatisch verschiedene Implementierungsalternativen 
an Kunden testen und auswerten. 
Womöglich lässt sich auch die Entscheidung für 
einen Cloud-Anbieter fundierter treffen, indem verglichen wird, wie gut die 
einzelnen Anbieter die Realisierung der Chancen und die Vermeidung der Risiken 
unterstützen. Dies bleibt jedoch in einer anderen Arbeit zu prüfen.

