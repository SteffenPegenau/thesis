%\addtocontents{toc}{\protect\newpage}
\section{Zusammenfassung und Ausblick}
\label{cha:fazit}
% Wohin sind wir gekommen
Zuletzt fassen Sie Ihre Arbeit kurz zusammen und stellen Ihre wichtigsten Schritte, Ergebnisse und Befunde dar. Geben Sie auch einen Ausblick auf mögliche anknüpfende Forschungsarbeiten. Außerdem findet sich hier Platz für eine kritische Hinterfragung einzelner Teilaspekte und auch für Ihre eigene Meinung.

\subsection{Offene Forschungsfragen}
\begin{itemize}
	\item Wie kann sich ein ISV auf die strukturellen Änderungen durch die 
Cloud-Migration bei seinen Kunden einstellen?
	\item Die identifizierten Chancen und Risiken hatten Auswirkungen im 
Bereich Visionsentwicklung/Strategie und der Anforderungsermittlung. Wie sind 
die anderen Phasen betroffen.
\end{itemize}




\subsection{Abgabedokument}
% Was wurde in der Arbeit alles gemacht
% Roten Faden aufgreifen
% TODO Pr�sens oder Pr�teritum
\textbf{Abschlussarbeiten} (Bachelor-, Master-, Diplomarbeit) sind in zweifacher 
Ausführung, einseitig bedruckt und gebunden abzugeben. Dazu auf CD die 
Abschlussarbeit in digitaler Form (z.B. Word und PDF), inkl. der 
Endnote-Projektdatei und der Grafiken. 
