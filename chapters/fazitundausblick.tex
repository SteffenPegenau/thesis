%\addtocontents{toc}{\protect\newpage}
\section{Zusammenfassung und Ausblick}
\label{cha:fazit}
% Wohin sind wir gekommen
Mittels einer Literaturanalyse konnte ein theoretisches Vorgehensmodell 
entwickelt werden, das ISV bei der Migration bestehender On-Premise-Anwendungen 
in die Cloud unterstützt, indem es Merkmale, Chancen und Risiken der Cloud 
identifiziert sowie ihre Wechselseitigen Einflüsse zu Geschäftsmodell und 
Strategie aufzeigt. 

Das entwickelte Vorgehensmodell wurde auf ein Projekt aus der Praxis 
angewendet, indem eine Cloud-Vision des bestehenden On-Premise-Produktes 
geschaffen wurde, die versucht Vorteile der Cloud weitestgehend zu realisieren 
und Risiken zu meiden. Um bei der Erstellung der Vision nicht den im 
Unternehmen gewohnten Trampelpfaden zu folgen, wurde an dieser Stelle zunächst 
bewusst darauf verzichtet, Meinungen und Erfahrungen einzuholen.

Dies geschah bei der abschließenden Evaluation des Modells, bei dem das von ihm 
vorgeschlagene mit dem tatsächlichen Vorgehen verglichen wurde. Dabei zeigte 
sich, dass das Modell geeignet ist, um eine Migration systematisch zu planen 
oder eine bestehende Planung zu analysieren. Objekt der Planung beziehungsweise 
der Analyse ist dabei nicht nur der Leistungsumfang der zu gestaltenden 
Cloud-Software sondern in mindestens ebenso großem Maße das Geschäftsmodell und 
die Unternehmensstrategie. Es wurde gezeigt, wie bedeutend die 
Wechselseitigen Einflüsse (Vgl. Abbildung~\ref{fig:wechselseitige_einfluesse}) 
sind, dass der Leistungsumfang an der unternehmerischen Strategie auszurichten 
ist und die Möglichkeiten der unternehmerischen Strategie von realisierbaren 
Merkmalen abhängen.
%\documentclass[border=10pt]{standalone}

\usetikzlibrary{decorations.text}
\usetikzlibrary{calc}
\usetikzlibrary{fit}
\usetikzlibrary{shapes}
\usetikzlibrary{arrows,positioning} 
\pgfmathsetmacro{\cubex}{4}
\pgfmathsetmacro{\cubey}{2}

\definecolor{light-gray}{gray}{0.80}

\tikzset{
    %Define standard arrow tip
    >=stealth',
    %Define style for boxes
    punkt/.style={
           rectangle,
           rounded corners,
           draw=black, very thick,
           text width=8em,
           minimum height=2em,
           
           text centered},
    % Define arrow style
    pil/.style={
           ->,
           very thick,
           %shorten <=2pt,
           shorten >=2pt,},
    sepLine/.style={
	dashed,
	very thick
    }
}


\tikzstyle{b} = [rectangle, draw, node distance=5cm, text 
width=9em, text centered, rounded corners, minimum height=4em, thick]
\tikzstyle{c} = [rectangle, draw, minimum height=15em, minimum width=10em, 
dashed]
\tikzstyle{l} = [draw,thick]

%\begin{document}
\begin{figure}[bh]
\begin{center}
\scalebox{1.0}{
\begin{tikzpicture}[auto]
    \node[b] (S) {Strategie};
    \node[b,below left of=ISV] (G) {Geschäftsmodell};
    \node[b,below right of=ISV] (M) {Cloud-Merkmale, -Chancen und -Risiken};
    
    
   \draw (S)
   edge[pil,<->] (G)
   edge[pil,<->] (M);
   
   \draw (M) edge[pil,<->] (G);
    
 
\end{tikzpicture}
}
\caption{Wechselseitige Einflüsse zwischen Strategie, Geschäftsmodell und 
Cloud-Merkmalen}
\label{fig:wechselseitige_einfluesse}
\end{center}
\end{figure}
Es erscheint vernünftig anzunehmen, dass der wirtschaftliche Erfolg von 
Projekten bei denen die genannten Wechselwirkungen bewusst bedacht wurden, als 
bei Projekten, bei denen dies nicht geschehen ist. Ein empirischer Beweis dafür 
muss allerdings noch erbracht werden. Ebenfalls überprüfenswert scheint die 
Vermutung, dass sich die Liste der Cloud-Merkmale dazu eignen könnte, passende 
Cloud-Anbieter zu identifizieren. Dies könnte geschehen, indem die 
Cloud-Anbieter darin vergleichen werden, wie gut sie den ISV oder ihre Kunden 
im Allgemeinen darin unterstützen, die identifizierten Chancen zu realisieren 
und die Risiken zu meiden. 

\begin{comment}
Zuletzt fassen Sie Ihre Arbeit kurz zusammen und stellen Ihre wichtigsten 
Schritte, Ergebnisse und Befunde dar. Geben Sie auch einen Ausblick auf mögliche 
anknüpfende Forschungsarbeiten. Außerdem findet sich hier Platz für eine 
kritische Hinterfragung einzelner Teilaspekte und auch für Ihre eigene Meinung.

\subsection{Offene Forschungsfragen}
\begin{itemize}
	\item Wie kann sich ein ISV und ein Softwarebetratungsunternehmen auf 
die strukturellen Änderungen durch die 
Cloud-Migration bei seinen Kunden einstellen?
	\item Die identifizierten Chancen und Risiken hatten Auswirkungen im 
Bereich Visionsentwicklung/Strategie und der Anforderungsermittlung. Wie sind 
die anderen Phasen betroffen.
	\item Einflüsse der Cloud auf Marketing, Strategie und Geschäftsmodell
\end{itemize}

\subsection{Schwierigkeiten}
Trennung zwischen Kundensicht und ISV-Sicht. Siehe Dreieck.




\subsection{Abgabedokument}
% Was wurde in der Arbeit alles gemacht
% Roten Faden aufgreifen
% TODO Pr�sens oder Pr�teritum
\textbf{Abschlussarbeiten} (Bachelor-, Master-, Diplomarbeit) sind in zweifacher 
Ausführung, einseitig bedruckt und gebunden abzugeben. Dazu auf CD die 
Abschlussarbeit in digitaler Form (z.B. Word und PDF), inkl. der 
Endnote-Projektdatei und der Grafiken. 
\end{comment}