%\addtocontents{toc}{\protect\newpage}
\section{Zusammenfassung und Ausblick}
\label{cha:fazit}
% Wohin sind wir gekommen
Mittels einer Literaturanalyse konnte ein theoretisches Vorgehensmodell 
entwickelt werden, das ISV bei der Migration bestehender On-Premise-Anwendungen 
in die Cloud unterstützt, indem es Merkmale, Chancen und Risiken der Cloud 
identifiziert sowie ihre Wechselseitigen Einflüsse zu Geschäftsmodell und 
Strategie aufzeigt. 

Das entwickelte Vorgehensmodell wurde auf ein Projekt aus der Praxis 
angewendet, indem eine Cloud-Vision des bestehenden On-Premise-Produktes 
geschaffen wurde, die versucht Vorteile der Cloud weitestgehend zu realisieren 
und Risiken zu meiden. Um bei der Erstellung der Vision nicht den im 
Unternehmen gewohnten Pfaden zu folgen, wurde an dieser Stelle zunächst 
bewusst darauf verzichtet, Meinungen und Erfahrungen einzuholen.

Dies geschah bei der abschließenden Evaluation des Modells, bei der die Vision 
der Umsetzung mit dem tatsächlichen Vorgehen verglichen wurde. Dabei zeigte 
sich, dass das Modell geeignet ist, um eine Migration systematisch zu planen 
oder eine bestehende Planung zu analysieren. Objekt der Planung beziehungsweise 
der Analyse ist dabei nicht nur der Leistungsumfang der zu gestaltenden 
Cloud-Software sondern in mindestens ebenso großem Maße das Geschäftsmodell und 
die Unternehmensstrategie. Es wurde gezeigt, wie bedeutend die 
Wechselseitigen Einflüsse (Vgl. Abbildung~\ref{fig:wechselseitige_einfluesse}) 
sind, dass der Leistungsumfang an der unternehmerischen Strategie auszurichten 
ist und die Möglichkeiten der unternehmerischen Strategie von realisierbaren 
Merkmalen abhängen.
%\documentclass[border=10pt]{standalone}

\usetikzlibrary{decorations.text}
\usetikzlibrary{calc}
\usetikzlibrary{fit}
\usetikzlibrary{shapes}
\usetikzlibrary{arrows,positioning} 
\pgfmathsetmacro{\cubex}{4}
\pgfmathsetmacro{\cubey}{2}

\definecolor{light-gray}{gray}{0.80}

\tikzset{
    %Define standard arrow tip
    >=stealth',
    %Define style for boxes
    punkt/.style={
           rectangle,
           rounded corners,
           draw=black, very thick,
           text width=8em,
           minimum height=2em,
           
           text centered},
    % Define arrow style
    pil/.style={
           ->,
           very thick,
           %shorten <=2pt,
           shorten >=2pt,},
    sepLine/.style={
	dashed,
	very thick
    }
}


\tikzstyle{b} = [rectangle, draw, node distance=5cm, text 
width=9em, text centered, rounded corners, minimum height=4em, thick]
\tikzstyle{c} = [rectangle, draw, minimum height=15em, minimum width=10em, 
dashed]
\tikzstyle{l} = [draw,thick]

%\begin{document}
\begin{figure}[bh]
\begin{center}
\scalebox{1.0}{
\begin{tikzpicture}[auto]
    \node[b] (S) {Strategie};
    \node[b,below left of=ISV] (G) {Geschäftsmodell};
    \node[b,below right of=ISV] (M) {Cloud-Merkmale, -Chancen und -Risiken};
    
    
   \draw (S)
   edge[pil,<->] (G)
   edge[pil,<->] (M);
   
   \draw (M) edge[pil,<->] (G);
    
 
\end{tikzpicture}
}
\caption{Wechselseitige Einflüsse zwischen Strategie, Geschäftsmodell und 
Cloud-Merkmalen}
\label{fig:wechselseitige_einfluesse}
\end{center}
\end{figure}
Es erscheint vernünftig anzunehmen, dass Migrationsprojekte bei denen 
systematisch Chancen und Risiken der Cloud bedacht wurden wirtschaftlich 
erfolgreicher sind als jene, bei denen dies nicht geschah. Die empirische 
Überprüfung dieser Annahme bleibt jedoch zu erbringen. Ebenfalls 
überprüfenswert scheint die Vermutung, dass sich die Liste der Cloud-Merkmale 
dazu eignen könnte, passende Cloud-Anbieter zu identifizieren. Dies könnte 
geschehen, indem die Cloud-Anbieter darin vergleichen werden, wie gut sie den 
ISV oder ihre Kunden im Allgemeinen darin unterstützen, die identifizierten 
Chancen zu realisieren und die Risiken zu meiden.

Thematisch lediglich am Rande gestreift wurden soziale und organisatorische 
Aspekte. Auch wenn Menschen, die beruflich einer Tätigkeit aus der IT nachgehen 
Kurzlebigkeit gewohnt sind, steigert sich die Geschwindigkeit des Wandels auf 
dem Weg in die Cloud erheblich. Es ist eine Frage an Projektleiter, wie die 
bereits hohe Geschwindigkeit und Agilität weiter gesteigert werden kann, um im 
Cloud Zeitalter weiter wettbewerbsfähig zu bleiben. Ebenfalls wirtschaftlich 
spannend ist die soziale Nachhaltigkeit: Wie lassen sich stabile Teams bilden, 
die die gestiegenen Anforderungen bewältigen können? Wie lassen sich dabei 
Überforderungen und Burnouts vermeiden? Gerade zu Teams historisch gewachsener, 
größerer Projekte dürften -- wie im Beispielprojekt dieser Arbeit -- 
Mitarbeiter gehören, die bisher wenig oder gar keine Erfahrungen mit der 
Cloud besitzen und sich nun im Wettbewerb mit einer sehr viel höheren 
Innovationsgeschwindigkeit sehen. Um weder diese Mitarbeiter noch ihr Know-How 
bei der Migration zu verlieren, müssen Gegenmaßnahmen getroffen 
werden. Wie diese Maßnahmen aussehen, muss in einer anderen Arbeit beantwortet 
werden.

Wenn Kritiker die Cloud als "`alten Wein in neuen Schläuchen"' 
\pcite{}{231}{softwareindustrie2015} beschreiben unterliegen sie einem 
gewaltigen Trugschluss: Die Cloud erhöht die Erwartungen, die Endanwender 
an Software haben in Hinblick auf Preis, Geschwindigkeit, Einfachheit, 
Flexibilität und Effizienz. Ihre ständige Verfügbarkeit wird die Gesellschaft 
immer weiter durchdringen. Softwarehersteller stehen vor der Herausforderung, 
den Wandel nicht nur in technischer, bilanzieller und strategischer Hinsicht zu 
meistern: Auch die Organisation und Kultur des Unternehmens muss überdacht 
werden.
\begin{comment}
Zuletzt fassen Sie Ihre Arbeit kurz zusammen und stellen Ihre wichtigsten 
Schritte, Ergebnisse und Befunde dar. Geben Sie auch einen Ausblick auf mögliche 
anknüpfende Forschungsarbeiten. Außerdem findet sich hier Platz für eine 
kritische Hinterfragung einzelner Teilaspekte und auch für Ihre eigene Meinung.

\subsection{Offene Forschungsfragen}
\begin{itemize}
	\item Wie kann sich ein ISV und ein Softwarebetratungsunternehmen auf 
die strukturellen Änderungen durch die 
Cloud-Migration bei seinen Kunden einstellen?
	\item Die identifizierten Chancen und Risiken hatten Auswirkungen im 
Bereich Visionsentwicklung/Strategie und der Anforderungsermittlung. Wie sind 
die anderen Phasen betroffen.
	\item Einflüsse der Cloud auf Marketing, Strategie und Geschäftsmodell
\end{itemize}

\subsection{Schwierigkeiten}
Trennung zwischen Kundensicht und ISV-Sicht. Siehe Dreieck.




\subsection{Abgabedokument}
% Was wurde in der Arbeit alles gemacht
% Roten Faden aufgreifen
% TODO Pr�sens oder Pr�teritum
\textbf{Abschlussarbeiten} (Bachelor-, Master-, Diplomarbeit) sind in zweifacher 
Ausführung, einseitig bedruckt und gebunden abzugeben. Dazu auf CD die 
Abschlussarbeit in digitaler Form (z.B. Word und PDF), inkl. der 
Endnote-Projektdatei und der Grafiken. 
\end{comment}