\section{Forschungsmethoden}
\label{cha:method}
Grundlage des in dieser Arbeit entwickelten Vorgehensmodells ist eine
systematische Literaturübersicht wie von \citeflow{kitchenham2004}
vorgeschlagen. Dabei werden zunächst für jede Forschungsfrage Schlüsselwörter
und ihre Synonyme identifiziert (Vgl. Tabelle
\ref{tab:forschungsfragen}) und mit booleschen Operatoren
verknüpft. Die entstandenen Ausdrücke (Vgl. Tabelle
\ref{tab:searchstrings}) dienten anschließend der Recherche in
Literaturdatenbanken (Vgl. Tabelle \ref{tab:literaturdatenbanken})

%
% WICHTIG
% Wenn hier eine Frage angepasst wird, muss sie auch in der Datei 
% forschungsfragen.tex angepasst werden
%
\begin{table}[h]
\centering
\begin{tabular}{|l|p{0.35\textwidth}|p{0.55\textwidth}|}
	\hline
	\textbf{\#} & \textbf{Frage} & \textbf{Rechercheausdruck} \\
	\hline
	1 & In welche Aufgaben lässt sich die Migration 
einer On-Premise-Software zu Salesforce unterteilen? & (tasks OR 
needs OR requirements)\newline AND\newline (migration OR adoption)\newline 
AND\newline salesforce
\\
	\hline
	2 & Welche Methoden unterstützen diesen Migrationsprozess? & (methods OR 
standards OR framework) \newline AND\newline 
('cloud migration' OR 'cloud adaption' OR 'salesforce') \\
	\hline
	3 & Wie unterstützt Salesforce die Migration technisch? & (tools OR 
interfaces OR api)\newline AND\newline (migration OR 
adoption)\newline 
AND\newline salesforce\\
	\hline
	4 & Wie wirkt sich die Migration auf die strategische Marktposition aus? 
& (strategy OR market) \newline 
AND\newline 
('cloud migration' OR 'cloud adaption') \\
	\hline
\end{tabular}
\caption{Forschungsfragen und zugehörige Rechercheausdrücke. Angelehnt 
an \cite{exploring_the_factors}
}
\label{tab:searchstrings}
\end{table}


Bei der systematischen Literaturübersicht wurden Ergebnisse berücksichtigt, die
\begin{itemize}
	\item den Rechercheausdrücken entsprachen
	\item in deutscher oder englischer Sprache vorlagen
	\item vollständig vorlagen
	\item in Abstract oder Fazit einen Zusammenhang zu den Forschungsfragen
aufwiesen
	\item seit einschließlich dem Jahr 2010 erschienen sind
\end{itemize}

Die fachliche Disziplin der Fragen beeinflusste die Wahl der
Literaturdatenbanken und ist in Tabelle \ref{tab:literaturdatenbanken}
dargestellt. So wurde beispielsweise darauf verzichtet in technischen
Datenbanken nach Literatur zu marktstrategischen Fragen zu suchen.

 \begin{table}[bh]
\centering
\begin{tabular}{|p{0.4\textwidth}|p{0.1\textwidth}|p{0.1\textwidth}|p{
0.1\textwidth}|}
\hline
  \hfill & \multicolumn{3}{c|}{\textbf{Forschungsfragen}} \\
  \hline
\textbf{Name und URL} & \textbf{1} & \textbf{2} & \textbf{3} \\
\hline
ACM Digital Library \newline \url{http://dl.acm.org/} & $\surd$ & ? & ? \\
	%Frage 2 \newline
	%\st{Frage 1,3,4}: Keine Volltextergebnisse \\
	\hline
	Science Direct \newline \url{http://www.sciencedirect.com/} & ?& 
?& ?\\
	\hline
	Wiley \newline \url{http://eu.wiley.com/} & \multicolumn{3}{c|}{Keine 
booleschen Ausdrücke möglich} \\
	%\st{Frage 1,2,3,4}: Keine Suche mit booleschen Ausdrücken möglich \\
	\hline
	Elektronische Zeitschriftenbibliothek (EZB)\newline
\url{http://rzblx1.uni-regensburg.de/ezeit/fl.phtml?bibid=TUDA} & 
\multicolumn{3}{c|}{Keine 
booleschen Ausdrücke möglich} \\
	\hline
	Compendex \newline
\url{https://www.elsevier.com/solutions/engineering-village/content/compendex} 
& \multicolumn{3}{c|}{Keine 
booleschen Ausdrücke möglich} \\
	\hline
	AIS Electronic Library (AISeL) \newline \url{http://aisel.aisnet.org/} & 
\multicolumn{3}{c|}{Keine 
booleschen Ausdrücke möglich} \\
	\hline
	Zeitschriftendatenbank (ZDB) \newline 
\url{http://dispatch.opac.ddb.de/DB=1.1/srt=YOP/} & ? & ? & ? \\
	\hline
	IEEE Xplore \newline 
	\url{http://ieeexplore.ieee.org/Xplore/dynhome.jsp?tag=1} & 
	? & ? & ? \\
	\hline
	Springer-Online: Bücher/Beiträge des Springer Verlags \newline
	\url{http://www.springerlink.com} & $\surd$ &  &  \\
	\hline
Rechercheangebot der ULB 
\newline \url{http://www.ulb.tu-darmstadt.de/recherche/} & $\surd$ & 
$\surd$ & $\surd$ \\
\hline
%	WiSo Net: deutschsprachige Literatur zu Wirtschafts- und 
%	Sozialwissenschaften \newline \url{www.wiso-net.de} & ? & ? & ? \\
%	\hline
	%EBSCO: internationale wirtschafts-wiss. Zeitschriften \newline 
	%\url{http://search.ebscohost.com} & ? & ? & ? \\
	%\hline
\end{tabular}

\caption{Literaturdatenbanken und für welche Fragen sie herangezogen wurden. 
Quellen: \cite{exploring_the_factors} und \cite{formatvorlage}}
\label{tab:literaturdatenbanken}
\end{table}
% $\surd$

\begin{comment}
\subsubsection{Sonstiges}
\begin{itemize}
\item \textbf{Google Scholar:} Suchdienst für wissenschaftliche Recherchen 
(http://scholar.google.de)
\item \textbf{Verlagswebseiten} Recherche und den Zugriff auf Zeitschriften- 
und 
Zeitungsartikel und E-Books
\item \textbf{Webseiten von Unternehmen} für die Recherche von 
Unternehmensdaten 
und-statistiken sowie Unternehmensdatenbanken
\item \textbf{Webseiten von Bundes- und Landesbehörden sowie der EU}
 Statistisches Bundesamt (http://www.destatis.de)
\\Presse- und Informationsamt der Bundesregierung 
(http://www.bundesregierung.de)
\item \textbf{Webseiten von Marktforschungsinstituten}
(für Marktanteile und Verbraucheranalysen)
\item \textbf{Webseiten von Verbänden und Kammern}
Institut der deutschen Wirtschaft (http://www.deutsche-wirtschaft.de)
\end{itemize}
\end{comment}


\subsection{TODO: Ideen für Recherchebegriffe}
\begin{itemize}
	\item Life Cycle
	\item Business Use Case
	\item strategy
	\item transition
	\item Cloudonomics
\end{itemize}


\begin{comment}
In diesem Kapitel erläutern Sie ihre Forschungsmethode unter Verwendung von
entsprechenden Quellen.
Begründen Sie auch, warum Sie sich für diese Forschungsmethode entschieden
haben
und warum sie geeignet ist, die vorliegende Forschungsfrage zu beantworten.
\end{comment}
