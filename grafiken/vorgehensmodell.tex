% PDCA cycle
% Author: tikzanfaenger, Helmut, and Bartman
\documentclass[tikz,border=10pt]{standalone}
\usetikzlibrary{decorations.text}
\usetikzlibrary{calc}
\definecolor{mygray}{RGB}{208,208,208}
\definecolor{mymagenta}{RGB}{226,0,116}

\pgfmathsetmacro{\cubex}{0.4cm}
\pgfmathsetmacro{\cubey}{0.15cm}
\pgfmathsetmacro{\cubez}{0.15cm}




\newcommand*{\mytextstyle}{\sffamily\Large\bfseries\color{black!85}}
\newcommand{\arcarrow}[3]{%
   % inner radius, middle radius, outer radius, start angle,
   % end angle, tip protusion angle, options, text
   \pgfmathsetmacro{\rin}{1.7}
   \pgfmathsetmacro{\rmid}{2.2}
   \pgfmathsetmacro{\rout}{2.7}
   \pgfmathsetmacro{\astart}{#1}
   \pgfmathsetmacro{\aend}{#2}
   \pgfmathsetmacro{\atip}{5}
   \fill[mygray, very thick] (\astart+\atip:\rin)
                         arc (\astart+\atip:\aend:\rin)
      -- (\aend-\atip:\rmid)
      -- (\aend:\rout)   arc (\aend:\astart+\atip:\rout)
      -- (\astart:\rmid) -- cycle;
   \path[
      decoration = {
         text along path,
         text = {|\mytextstyle|#3},
         text align = {align = center},
         raise = -1.0ex
      },
      decorate
   ](\astart+\atip:\rmid) arc (\astart+\atip:\aend+\atip:\rmid);
}


\newcommand{\cube}[1]{
\definecolor{front-gray}{gray}{0.80}
\definecolor{top-gray}{gray}{0.9}
\definecolor{right-gray}{gray}{0.9}

\draw[black,fill=front-gray] (0,0,0) -- ++(-\cubex,0,0) -- ++(0,-\cubey,0) -- 
++(\cubex,0,0) -- cycle;
\draw[black,fill=right-gray] (0,0,0) -- ++(0,0,-\cubez) -- ++(0,-\cubey,0) -- 
++(0,0,\cubez) -- cycle;
\draw[black,fill=top-gray] (0,0,0) -- ++(-\cubex,0,0) -- ++(0,0,-\cubez) -- 
++(\cubex,0,0) -- cycle;
\draw (-0.5*\cubex,-0.5*\cubey) node {\Huge #1};
}

\newcommand{\trichter}{
	\newcommand{\hObenMitte}{4.5cm}
	\newcommand{\hMitteUnten}{4cm}
	\newcommand{\ellipsenFaktor}{0.1}
	\newcommand{\rObenX}{7.5cm}
	\newcommand{\rObenY}{\ellipsenFaktor*\rObenX}
	\newcommand{\rMitteX}{3cm}
	\newcommand{\rMitteY}{\ellipsenFaktor*\rMitteX}

	\coordinate (obenCenter) at (-0.4*\cubex,-1.4*\cubey);
	\coordinate (obenLinks) at ($ (obenCenter) + (-\rObenX,0) $);
	\coordinate (obenRechts) at ($ (obenCenter) + (+\rObenX,0) $);

	\coordinate (mitteCenter) at ($ (obenCenter) + (0,-\hObenMitte) $);
	\coordinate (mitteLinks) at ($ (mitteCenter) + (-\rMitteX,0) $);
	\coordinate (mitteRechts) at ($ (mitteCenter) + (+\rMitteX,0) $);

	\coordinate (untenCenter) at ($ (mitteCenter) + (0,-\hMitteUnten) $);
	\coordinate (untenLinks) at ($ (untenCenter) + (-\rMitteX,0) $);
	\coordinate (untenRechts) at ($ (untenCenter) + (+\rMitteX,0) $);

	\draw (obenCenter) ellipse[x radius=\rObenX,y radius=\rObenY];
	\draw (mitteCenter) ellipse[x radius=\rMitteX,y radius=\rMitteY];
	\draw (untenCenter) ellipse[x radius=\rMitteX,y radius=\rMitteY];

	\draw (obenLinks) -- (mitteLinks);
	\draw (obenRechts) -- (mitteRechts);

	\draw (mitteLinks) -- (untenLinks);
	\draw (mitteRechts) -- (untenRechts);
}

\begin{document}



\begin{tikzpicture}[join=round]

\cube{On-Premise-L\"osung}
\trichter



\begin{scope}[xshift=-11*\cubex,yshift=-104\cubey,scale=1.5] %52y
   \tikzstyle{conefill} = [fill=blue!20,fill opacity=1]
    \tikzstyle{ann} = [fill=white,font=\footnotesize,inner sep=1pt]
    \tikzstyle{ghostfill} = [fill=white]
    \tikzstyle{ghostdraw} = [draw=black!50]
    \filldraw[conefill](-.775,1.922)--(-1.162,.283)--(-.274,.5)
                        --(-.183,2.067)--cycle;
    \filldraw[conefill](-.183,2.067)--(-.274,.5)--(.775,.424)
                        --(.516,2.016)--cycle;
    \filldraw[conefill](.516,2.016)--(.775,.424)--(1.369,.1)
                        --(.913,1.8)--cycle;
    \filldraw[conefill](-.913,1.667)--(-1.369,-.1)--(-1.162,.283)
                        --(-.775,1.922)--cycle;
    \filldraw[conefill](.913,1.8)--(1.369,.1)--(1.162,-.283)
                        --(.775,1.545)--cycle;
    \filldraw[conefill](-.516,1.45)--(-.775,-.424)--(-1.369,-.1)
                        --(-.913,1.667)--cycle;
    \filldraw[conefill](.775,1.545)--(1.162,-.283)--(.274,-.5)
                        --(.183,1.4)--cycle;
    \filldraw[conefill](.183,1.4)--(.274,-.5)--(-.775,-.424)
                        --(-.516,1.45)--cycle;
    \node[ann] at (-.456,2.307) {Vision};
\end{scope}
\end{tikzpicture}
\end{document}