%\documentclass[border=10pt]{standalone}

\usetikzlibrary{decorations.text}
\usetikzlibrary{calc}
\usetikzlibrary{fit}
\usetikzlibrary{shapes}
\usetikzlibrary{arrows,positioning} 
\pgfmathsetmacro{\cubex}{4}
\pgfmathsetmacro{\cubey}{2}

\definecolor{light-gray}{gray}{0.80}

\tikzset{
    %Define standard arrow tip
    >=stealth',
    %Define style for boxes
    punkt/.style={
           rectangle,
           rounded corners,
           draw=black, very thick,
           text width=8em,
           minimum height=2em,
           text centered},
    % Define arrow style
    pil/.style={
           ->,
           very thick,
           shorten <=5pt,
           shorten >=5pt,},
    sepLine/.style={
	dashed,
	very thick
    }
}




%\begin{document}
\begin{figure}[bh]
\begin{center}
\scalebox{.8}{
\begin{tikzpicture}


\newcommand{\gear}[5]{%
    \foreach \i in {1,...,#1}
    {   [rotate=(\i-1)*360/#1] (0:#2) arc (0:#4:#2) {[rounded corners=0.5pt] -- 
(#4+#5:#3)  arc (#4+#5:360/#1-#5:#3)} --  (360/#1:#2)
    }
}      

% Punkte
\newcommand{\boxNO}{(0,0)}
\newcommand{\boxNW}{(-\cubex,0)}
\newcommand{\boxSO}{(0,-\cubey)}
\newcommand{\boxSW}{(-\cubex,-\cubey)}


% On-Premise-Produkt
\draw[black,fill=light-gray,very thick] \boxNW -- \boxNO -- \boxSO -- \boxSW -- 
cycle;
\draw (-0.5*\cubex,-0.5*\cubey) node {\textbf{On-Premise-Produkt}};

% Trichter
\newcommand{\trichterBreiteO}{1.5*\cubex}
\newcommand{\trichterBreiteM}{0.33*\cubex}
\newcommand{\trichterAbstand}{0.5*(\trichterBreite - \trichterBreiteM)}

\newcommand{\trX}{-1.25*\cubex}
\newcommand{\trY}{-1.25*\cubey}
\newcommand{\trichterNW}{ (\trX, \trY) }
\newcommand{\trichterNO}{ (\trX+\trichterBreiteO, \trY) }

\newcommand{\trichterMW}{ ( \trX + 0.33*\trichterBreiteO, \trY-0.75*\cubey ) }
\newcommand{\trichterMO}{ ( \trX + 0.66*\trichterBreiteO, \trY-0.75*\cubey ) }

\newcommand{\trichterSW}{ ( \trX + 0.33*\trichterBreiteO, \trY-1.75*\cubey) }
\newcommand{\trichterSO}{ ( \trX + 0.66*\trichterBreiteO, \trY-1.75*\cubey) }

\newcommand{\trichterC}{ ( \trX + 0.5*\trichterBreiteO, \trY-0.5*0.75*\cubey) }

\draw[black,thick] %,fill=SkyBlue 
	\trichterNW -- 
	\trichterNO -- 
	\trichterMO --
	\trichterSO --
	\trichterSW --
	\trichterMW --
	\trichterNW;
% Uncomment for text in Trichter
%\draw \trichterC node {};




% Diamond



\newcommand{\diamondX}{\trX + 0.5*\trichterBreiteO}
\newcommand{\diamondY}{\trY-2.4*\cubey}
%star point height=.9cm, minimum size=0.7*\cubey,draw,
\node [starburst, fill=Goldenrod,draw,starburst points=13] (VISION)
at (\diamondX,\diamondY) {Vision};
	
\newcommand{\boxREx}{\diamondX}
\newcommand{\boxREy}{\diamondY - 1.5*\cubey}
\newcommand{\boxRo}{1.8*\cubey}
\newcommand{\boxRm}{2*\cubey}
\newcommand{\boxRu}{1.2*\cubey}
\newcommand{\boxHEins}{.9*\cubey}
\newcommand{\boxHZwei}{2*\cubey}
\newcommand{\boxHDrei}{3*\cubey}


 \node[punkt] (RE) at (\boxREx,\boxREy) {Requirements Engineering} ;  
\node[punkt] (PL) at (\boxREx-\boxRo,\boxREy -\boxHEins)  {Planning}
 edge[pil] (RE.west) ;
 
 \node[punkt] (EVAL) at (\boxREx-\boxRm,\boxREy -\boxHZwei)  {Evaluation} 
edge[pil] (PL.south);
 
 
 \node[punkt] (T) at (\boxREx-\boxRu,\boxREy -\boxHDrei)  {Testing}
 edge[pil] (EVAL.south) ;
 
 
 \node[punkt] (DEPL) at (\boxREx+\boxRu,\boxREy -\boxHDrei)  {Deployment}
 edge[pil] (T.east);
 
 
 \node[punkt] (I) at (\boxREx+\boxRm,\boxREy -\boxHZwei)  {Implementation}
 edge[pil] (DEPL.north);
 
 \node[punkt] (AD) at (\boxREx+\boxRo,\boxREy -\boxHEins)  {Analysis \& Design}
edge[pil] (I.north);

\draw[pil]
  (RE.east) edge (AD.north);
  
% Pfeil+Text Vision -> RE
\draw[pil] (VISION) edge (RE.north);% {Wie sieht es hier aus?};

%\node (UMSETZBARKEIT) at (\diamondX+0.25*\cubex,\diamondY-0.6*\cubey) 
%{\Large{1001 ?}};
 %(formidler) {Intermediaries (c)};
 
% given
\pgfmathsetmacro{\rOne}{0.2}% inner radius
\pgfmathsetmacro{\nOne}{11}% num teeth
\pgfmathsetmacro{\nTwo}{15}
\pgfmathsetmacro{\nThree}{19}
\pgfmathsetmacro{\toothHeight}{0.4*\rOne}
\pgfmathsetmacro{\toothFall}{1}% degrees where radius drops from inner 
to %outer
\pgfmathsetmacro{\aOneTwo}{120}% angle from first to second
\pgfmathsetmacro{\aTwoThree}{40}% angle from first to second

% computed
\pgfmathsetmacro{\rTwo}{\rOne*\nTwo/\nOne}% via equating tooth lengths
\pgfmathsetmacro{\rThree}{\rOne*\nThree/\nOne}% 
\pgfmathsetmacro{\ROne}{\rOne+\toothHeight}% outer radius
\pgfmathsetmacro{\RTwo}{\rTwo+\toothHeight}
\pgfmathsetmacro{\RThree}{\rThree+\toothHeight}
\pgfmathsetmacro{\dOne}{(360/\nOne-\toothFall)/2}% degrees for a tooth; here 
inner deg = outer deg
\pgfmathsetmacro{\dTwo}{(360/\nTwo-\toothFall)/2}
\pgfmathsetmacro{\dThree}{(360/\nThree-\toothFall)/2}
\pgfmathsetmacro{\distOneTwo}{\rOne+\rTwo+\toothHeight+0.05}
\pgfmathsetmacro{\distTwoThree}{\rTwo+\rThree+\toothHeight+0.05}

\coordinate (A) at (\diamondX+0.5*\cubex,\diamondY-0.8*\cubey);
\coordinate (B) at ($(A)+(\aOneTwo:\distOneTwo)$);
\coordinate (C) at ($(B)+(\aTwoThree:\distTwoThree)$);  

\draw[thick, rotate=10] node[align=center] {} ;
%\gear{\nOne}{\rOne}{\ROne}{\dOne}{\toothFall} ;

\draw[thick, shift={(B)}, rotate=18] node[align=center] {} 
\gear{\nTwo}{\rTwo}{\RTwo}{\dTwo}{\toothFall};
\draw[thick, shift={(C)}, rotate=8] node[align=center] {} 
\gear{\nThree}{\rThree}{\RThree}{\dThree}{\toothFall};

\node[scale=1.5] (TECHQ) at ($ (A) + (0.3*\cubex,0.4*\cubey) $) {\Huge{?}};
\node[scale=1.5] (WIRTQ) at (\diamondX-0.6*\cubex,\diamondY-0.4*\cubey) 
{\Huge{€?}};
 
% Cloud 
\node [cloud, draw,cloud puffs=10,cloud puff arc=120, aspect=2, inner 
ysep=3*\cubex,fill=Goldenrod] (CLOUD) at (\boxREx,\boxREy-0.8*\boxHZwei) 
{\textbf{Cloud L\"osung}};
 
% Linien
\newcommand{\firstLineY}{\trY - 1.9 * \cubey}
\draw[sepLine] (\trX-\boxRm, \firstLineY) -- (\trX + 2.3*\boxRm, 
\firstLineY);

\newcommand{\secondLineY}{\diamondY - .85 * \cubey}
\draw[sepLine] (\trX-\boxRm, \secondLineY) -- (\trX + 2.3*\boxRm, 
\secondLineY);

% Phasen
\newcommand{\circlesX}{\diamondX-1.5*\boxRm}
\draw (\circlesX,\diamondY + 1.2*\cubey) circle [radius=0.4*\cubey] node 
{\Huge{I}};
\draw (\circlesX,\diamondY - 0.2*\cubey) circle [radius=0.4*\cubey] node 
{\Huge{II}};
\draw (\circlesX,\diamondY - 1.5*\cubey) circle [radius=0.4*\cubey] node 
{\Huge{III}};
 
 %\node[punkt, inner sep=5pt,below=0.5cm of RE]
 %(formidler) {Intermediaries (c)};
 % We make a dummy figure to make everything look nice.
 %\node[above=of RE] (dummy) {};
 %\node[right=of dummy] (t) {Ultimate borrower}
  %edge[pil,bend left=45] (RE.east) % edges are used to connect two nodes
  % edge[pil, bend left=45] (I.east); % .east since we want
                                             % consistent style
 %\node[left=of dummy] (g) {Ultimate lender}
   %edge[pil, bend right=45] (RE.west)
   %edge[pil,<->, bend left=45] node[auto] {Direct (a)} (t);
	
\end{tikzpicture}
}
\caption{Darstellung des Vorgehensmodells. Der iterative Software Engineering 
Kreislauf wurde \cite{changes_in_requirements_engineering} entnommen.}
\label{fig:vorgehensmodell}
\end{center}
\end{figure}

%\end{document}


