\usetikzlibrary{arrows}
\begin{figure}[bh]
\begin{center}
\begin{tikzpicture}
\newcommand{\rechts}{6}
\newcommand{\radius}{1.9cm}
\coordinate (centerTop) at (\rechts,3.5);
\coordinate (centerLinks) at (0,0);
\coordinate (centerRechts) at (\rechts,0);
\coordinate (centerBottom) at (\rechts,-1);

% Linker Kreis
\draw (centerLinks) circle (\radius) node [text=black] 
{On-Premise-Lösung};


% Oberer Kreis
\draw[fill=green] (centerTop) circle (\radius);
\node[align=left] at ($ (centerTop) + (3.5,0)$) {Chancen};

%Unterer Kreis mit Schrift
\draw[fill=red] (centerBottom) circle (\radius);
\node[align=left] at ($ (centerBottom) + (3.5,-1)$) {Risiken};


% Rechte Mitte
\draw (centerRechts) circle (\radius);
\node[align=left] at ($ (centerRechts) + (3.5,0)$) 
{\st{On-Premise-}\\Cloud-Lösung};

% Pfeil
\path[draw=black,solid,line width=1mm,fill=black,
preaction={-triangle 90,thin,draw,shorten >=-1mm}
] ($ (centerLinks) + (1.2 * \radius,0) $) -- ($ (centerRechts) + 
(-1.2 * \radius,0) $);

%\draw \secondcircle node [text=black,left] {On-Premise-Software};
%\draw \thirdcircle node [text=black,right] {$C$};
\end{tikzpicture}
\caption{Funktionsumfang vor und nach der Migration im Wasserfallmodell. 
Selbsterstellte Grafik.}
\label{fig:funktionsumfang_wasserfallmodell}
\end{center}
\end{figure}
