\begin{description}
	\cloudFeature{Hohe Kosten durch Pay-per-Use}{Ist das Bezahlmodell ein 
"`Pay-per-Use"'-Modell können bei rechen- oder speicherintensiven Anwendungen 
sehr hohe Kosten entstehen. Dies kann -- zum Beispiel bei einem 
Online-Shop -- gewollt sein, da die Anwendung bei hoher Last zwar teurer 
wird aber bedienbar bleibt, was gut für die Kundenzufriedenheit ist und dabei 
hohe Umsätze generiert. Wissenschaftliche Berechnungen hingegen sind ein 
Beispiel, bei dem eine gewisse Wartezeit auf die Ergebnisse vielleicht in Kauf 
genommen worden wäre, um explodierende Kosten zu 
vermeiden. Dieses Problem betrifft zwar 
häufig in erster Linie den Kunden des 
ISV. Im Sinne der Kundenzufriedenheit 
sollte dieser Aspekt dennoch vom ISV 
bedacht werden.}{challenges_and_assessment_in_migrating, 
migrating_to_or_away_from_the_public_cloud}{Wie lässt 
sich der Leistungsbedarf der Anwendung charakterisieren? Ist die 
Bereitstellung der angeforderten Leistung auch bei höheren Kosten für 
das Unternehmen lukrativ? Bietet der Cloud-Anbieter eine Deckelung der Leistung 
an, um explodierende Kosten zu vermeiden?
}
	\cloudFeature{Fehlende Interoperabilität / Komplexität 
gekoppelter Komponenten}{Gerade bei Anbindung eigener Anwendungen ist 
Interoperabilität wichtig. Cloud-Dienste 
sollten in die bestehende Struktur integriert werden. Darauf ist bei der 
Integration zu achten, da das System mit der steigenden Zahl genutzter Dienste 
bei verschiedenen Anbietern komplexer wird. Deshalb sollte die Verbindung 
zwischen Cloud Diensten, aber auch die Verbindung von On-Premise-Anwendungen 
mit der Cloud langfristig gedacht 
werden.}{towards_modelling_a_cloud_applications_life_cycle, 
cloud_based_next_generation_service_and_key_challenges, 
challenges_and_assessment_in_migrating, 
towards_an_understanding_of_cloud_computings_impact_on_org_it_strategy,
cloudward_bound_planning_for_beneficial_migration}{Welche eigenen Systeme und 
Anwendungen müssen an die Cloud-Anwendungen gekoppelt werden? Welche 
Cloud-Plattform unterstützt diese Anbindung am besten? Welche
Kopplungen sind absehbar? Wie sieht eine Architektur aus die (absehbare)
Kopplungen begünstigt?}

	\cloudFeature{Lock-in Effekte}{Der ISV kann von zwei Seiten vom
Lock-in-Effekt betroffen sein:
Zum einen auf Plattformen, die er nutzt um für Kunden Software zu entwickeln. Zum anderen 
von Lock-in-Effekten, die der ISV seinen Kunden
aufbürdet.}{disruptive_technologies_a_business_model_perspective,
challenges_and_assessment_in_migrating,
migrating_to_or_away_from_the_public_cloud}{Wie 
umfangreich ist der Lock-in-Effekt auf beiden Seiten? Wie lässt er sich minimieren?}

	\cloudFeature{Datenmigration}{Zeitgleiche Datenmigration und der parallele
Betrieb mit der alten Lösung können Dateninkonsistenzen hervorrufen.}
{towards_an_understanding_of_cloud_computings_impact_on_org_it_strategy}{Welche 
Daten müssen in welchen Mengen migriert werden? Wie lässt sich ein 
reibungsloser Ablauf organisieren?}

\cloudFeature{Leistungstransparenz}{Da der Kunde die Cloud-Lösung in der Regel vor
Vertragsabschluss ausprobieren kann, muss der Anbieter einen echten Mehrwert
bieten.}{how_saas_changes_an_isvs_business}{Wie lässt sich die Bedienung visuell ansprechend, aber auch 
unkompliziert gestalten, um einen guten ersten Eindruck beim Kunden zu 
hinterlassen? Welche Features sorgen für Begeisterung und sollten besonders 
prominent platziert werden?}

\cloudFeature{Geringere Umsätze}{Umsätze kommen zwar stetig aber in viel geringerer Höhe als bei
einem Kauf.}{how_saas_changes_an_isvs_business}{Welche Verkaufszahlen sind 
nötig um auch mit geringerem Umsatz profitabel zu sein? Wurden die Shareholder 
von der neuen Umsatzstrategie überzeugt?}

\cloudFeature{Geringere Anpassbarkeit}{Durch die in der Cloud übliche 
Standardisierung kann die Anpassbarkeit 
leiden.}{how_saas_changes_an_isvs_business}{Bietet der gewählte Cloud-Provider 
die nötige Anpassbarkeit? Bei welchen Komponenten lässt sich auf 
standardisierte Lösungen zurückgreifen und welche erfordern eigene 
Entwicklungen?}

\cloudFeature{Organisatorische/Strukturelle Umbrüche}{Der Wechsel in der Cloud erfordert viele neue Ansätze, zum
Beispiel bei der Lizenzierung oder dem Verkauf und stellt einen großen Umbruch 
dar. }{how_saas_changes_an_isvs_business,
towards_an_understanding_of_cloud_computings_impact_on_org_it_strategy}{Wie 
lässt sich die Organisation auf die höhere Geschwindigkeit zu 
möglicherweise geringeren Kosten umstellen? Wie lassen sich die neuen 
organisatorischen Anforderungen möglichst gut im Unternehmen umsetzen?}

\cloudFeature{Updatefrequenz erfordert Agilität}{SaaS-Anwendungen werden 
typischerweise sehr viel häufiger geupdatet, als
On-Premise-Anwendungen. Die hohe Updatefrequenz macht agile Entwicklungsmodelle 
erforderlich.}{how_saas_changes_an_isvs_business}{Inwiefern müssen bestehende 
(agile) Entwicklungsprozesse angepasst werden? Wie lässt sich eine stetige 
Verbesserung organisieren?}
\end{description}
