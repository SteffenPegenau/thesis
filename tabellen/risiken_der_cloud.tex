\begin{description}
	\cloudFeature{Hohe Kosten durch Pay-per-Use}{Ist das Bezahlmodell ein 
"`Pay-per-Use"'-Modell können bei rechen- oder speicherintensiven Anwendungen 
sehr hohe Kosten entstehen. Dies kann -- zum Beispiel bei einem 
Online-Shop -- gewollt sein, da die Anwendung bei hoher Last zwar teurer 
wird aber bedienbar bleibt, was gut für die Kundenzufriedenheit ist und dabei 
hohe Umsätze generiert. Wissenschaftliche Berechnung hingegen sind ein 
Beispiel, bei dem eine gewisse Wartezeit auf die Ergebnisse vielleicht in Kauf 
genommen worden wäre, um explodierende Kosten zu 
vermeiden.}{challenges_and_assessment_in_migrating,
migrating_to_or_away_from_the_public_cloud}{Wie lässt 
sich der Leistungsbedarf der Anwendung charakterisieren? Ist die 
Bereitstellung der angeforderten Leistung auch bei höheren Kosten für 
das Unternehmen lukrativ? Bietet der Cloud-Anbieter eine Deckelung der Leistung 
an, um explodierende Kosten zu vermeiden?
}

	\cloudFeature{Fehlende Interoperabilität}{Gerade bei Anbindung eigener 
Anwendungen ist Interoperabilität 
wichtig. Cloud-Dienste 
sollten in das bestehende Business integriert werden. Darauf ist auf die 
Integration zu achten, da das System mit der steigenden Zahl genutzter Dienste 
bei verschiedenen Anbietern komplexer wird. Deshalb sollte die Verbindung 
zwischen Cloud Diensten, aber auch die Verbindung von On-Premise-Anwendungen 
mit 
der Cloud langfristig gedacht 
werden.}{towards_modelling_a_cloud_applications_life_cycle, 
cloud_based_next_generation_service_and_key_challenges, 
challenges_and_assessment_in_migrating,
towards_an_understanding_of_cloud_computings_impact_on_org_it_strategy}{Welche 
eigenen Systeme und Anwendungen müssen an die Cloud-Anwendungen gekoppelt 
werden? Welche Cloud-Plattform unterstützt diese Anbindung am besten?}

	\cloudFeature{Lock-in Effekte}{Der ISV kann von zwei Seiten vom
Lock-in-Effekt betroffen sein:
1. Auf Plattformen, die er nutzt um für Kunden Software zu entwickeln. 2.
Von Lock-in-Effekten, die der ISV seinen Kunden
aufbürdet.}{disruptive_technologies_a_business_model_perspective,
challenges_and_assessment_in_migrating,
migrating_to_or_away_from_the_public_cloud}{Wie 
umfangreich ist der Lock-in auf beiden Seiten? Wie lässt er sich minimieren?}

	\cloudFeature{Komplexität unbedacht gekoppelter Komponenten}{Auch
wenn die Ankopplung standardisierter Cloud-Komponenten relativ leicht ist,
erhöht sich mit jeder Komponente die
Komplexität}{cloud_based_next_generation_service_and_key_challenges,
cloudward_bound_planning_for_beneficial_migration,
challenges_and_assessment_in_migrating}{Welche
Kopplungen sind absehbar? Wie sieht eine Architektur aus die (absehbare)
Kopplungen begünstigt?}

	\cloudFeature{Datenmigration}{Datenmigrationen oder der parallele
Betrieb mit der alten Lösung können Inkonsistenzen in den Daten hervorrufen}
{towards_an_understanding_of_cloud_computings_impact_on_org_it_strategy}{}

\cloudFeature{Leistungstransparenz}{Da der Kunde die Cloud-Lösung in der Regel vor
Vertragsabschluss ausprobieren kann, muss der Anbieter einen echten Wert
bieten.}{how_saas_changes_an_isvs_business}{}

\cloudFeature{Geringere Umsätze}{Umsätze kommen zwar stetig aber in viel geringerer Höhe als bei
einem Kauf.}{how_saas_changes_an_isvs_business}{}

\cloudFeature{Geringere Anpassbarkeit}{Durch die in der Cloud übliche Standardisierung kann die Anpassbarkeit leiden}{how_saas_changes_an_isvs_business}{}

\cloudFeature{Organisatorische/Strukturelle Umbrüche}{Der Wechsel in der Cloud erfordert viele neue Ansätze, zum
Beispiel bei der Lizenzierung oder dem Verkauf und stellt einen großen Umbruch 
dar. Höhere Entwicklungs- und  
Deploymentgeschwindigkeit mit weniger involvierten Menschen, geringeren Kosten 
und besserer 
Performance.}{how_saas_changes_an_isvs_business,
towards_an_understanding_of_cloud_computings_impact_on_org_it_strategy}{}

\cloudFeature{Updatefrequenz erfordert Agilität}{Da SaaS-Anwendungen werden 
typischerweise sehr viel häufiger geupdatet, als
On-Premise-Anwendungen. Die hohe Updatefrequenz macht agile Entwicklungsmodelle erforderlich}{how_saas_changes_an_isvs_business}{}
\end{description}
