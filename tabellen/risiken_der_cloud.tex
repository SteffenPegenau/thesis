\begin{description}
	\cloudFeature{Hohe Kosten durch Pay-per-Use}{Die migrierte Software ist durch die für sie erforderliche dauerhafte Hochlast sehr viel teurer}{}{}

	\cloudFeature{Lock-in Effekte}{Der ISV ist von zwei Seiten vom
Lock-in-Effekt betroffen:
1. Auf Plattformen, die er nutzt um für Kunden Software zu entwickeln. 2.
Lock-in-Effekte, die der ISV seinen Kunden
aufbürdet.}{disruptive_technologies_a_business_model_perspective}{}

	\cloudFeature{Komplexität unbedacht gekoppelter Komponenten}{Auch
wenn die Ankopplung standardisierter Cloud-Komponenten relativ leicht ist,
erhöht sich mit jeder Komponente die
Komplexität}{cloud_based_next_generation_service_and_key_challenges}{Welche
Kopplungen sind absehbar? Wie sieht eine Architektur aus die (absehbare)
Kopplungen begünstigt?}

	\cloudFeature{Datenmigration}{Datenmigrationen oder der parallele
Betrieb mit der alten Lösung können Inkonsistenzen in den Daten hervorrufen}
{towards_an_understanding_of_cloud_computings_impact_on_org_it_strategy}{}

\cloudFeature{Leistungstransparenz}{Da der Kunde die Cloud-Lösung in der Regel vor
Vertragsabschluss ausprobieren kann, muss der Anbieter einen echten Wert
bieten.}{how_saas_changes_an_isvs_business}{}

\cloudFeature{Geringere Umsätze}{Umsätze kommen zwar stetig aber in viel geringerer Höhe als bei
einem Kauf.}{how_saas_changes_an_isvs_business}{}

\cloudFeature{Geringere Anpassbarkeit}{Durch die in der Cloud übliche Standardisierung kann die Anpassbarkeit leiden}{how_saas_changes_an_isvs_business}{}

\cloudFeature{Organisatorische/Strukturelle Umbrüche}{Der Wechsel in der Cloud erfordert viele neue Ansätze, zum
Beispiel bei der Lizenzierung oder dem Verkauf und stellt einen großen Umbruch dar. }{how_saas_changes_an_isvs_business}{}

\cloudFeature{Updatefrequenz erfordert Agilität}{Da SaaS-Anwendungen werden 
typischerweise sehr viel häufiger geupdatet, als
On-Premise-Anwendungen. Die hohe Updatefrequenz macht agile Entwicklungsmodelle erforderlich}{how_saas_changes_an_isvs_business}{}
\end{description}
