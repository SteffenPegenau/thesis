\begin{table}[ht!]
\centering
\begin{longtable}{|p{0.45\textwidth}|p{0.45\textwidth}|}
\hline
\textbf{On-Premise-Software} & \textbf{Software as a Service} \\
\hline %%%%%%%%%%%%%%%%%%%%%%%%%%%%%%%%%%%%%%%%%%%%%%%%%%%%%%%%%%%%%%%%%
Überschaubare Anzahl von Stakeholdern & Viele verschiedene Stakeholder \\
\hline
Keine oder geringe Einbeziehung des Kunden in die Entwicklung & Starke 
Einbeziehung des Kunden in die Entwicklung\\ \hline
Geschäftsbeziehung zum Kunden endet mit einmaliger Zahlung & langfristige 
Geschäftsbeziehung \\ \hline
Nutzungserfahrungen nur über spezielle Erhebungen & Direkte Rückmeldungen durch die Kunden, ggf.
motiviert durch die Hoffnung auf Fehlerbehebung oder neue Features \\ \hline
Regelmäßige, geplante Fehlerbehebung & Sofortige Fehlerbehebung \\ \hline
Keine neuen Features ohne Versionsupgrade & Andauernde Auslieferung neuer 
Features, ohne größere Verzögerung \\ \hline
Updates und Upgrades erfordern Downtime & Nahtloser Updateprozess ohne 
Unterbrechung \\ \hline
Upgrades haben größere Auswirkungen und machen Schulungen erforderlich & 
Kontinuierliche Auslieferung weniger disruptiv \\ \hline
Prognose und Test der Akzeptanz schwierig & Alternativen lassen sich an 
kleineren Nutzergruppen testen \\
\hline %%%%%%%%%%%%%%%%%%%%%%%%%%%%%%%%%%%%%%%%%%%%%%%%%%%%%%%%%%%%%%%%%
\end{longtable}
\caption{Unterschiede im Requirementsengineering. Entnommen aus 
\cite{changes_in_requirements_engineering}}
\label{tab:unterschiede_im_re}
\end{table}