 \begin{table}[h]
 
\newcommand{\fSmall}{0.05}
\centering
\begin{tabular}
{|p{0.7\textwidth}|p{0.05\textwidth}|p{0.05\textwidth}|p{0.05\textwidth}|}
\hline
   & \multicolumn{3}{p{0.15\textwidth}|}{\textbf{Forschungsfragen}} \\
  \hline
\textbf{Name und URL} & \textbf{1} & \textbf{2} & \textbf{3} \\
\hline
ACM Digital Library \newline \url{http://dl.acm.org/} & $\surd$ & ? & ? \\
	%Frage 2 \newline
	%\st{Frage 1,3,4}: Keine Volltextergebnisse \\
	\hline
	Science Direct \newline \url{http://www.sciencedirect.com/} & ?& 
?& ?\\
	\hline Wiley \newline \url{http://eu.wiley.com/} & 
\multicolumn{3}{p{0.2\textwidth}|}{Keine booleschen\newline Ausdrücke möglich} 
\\
	%\st{Frage 1,2,3,4}: Keine Suche mit booleschen Ausdrücken möglich \\
	\hline
	Elektronische Zeitschriftenbibliothek (EZB)\newline
\url{http://rzblx1.uni-regensburg.de/ezeit/fl.phtml?bibid=TUDA} & 
\multicolumn{3}{p{0.21\textwidth}|}{Keine booleschen\newline Ausdrücke möglich} 
\\
	\hline
	Compendex \newline
\url{https://www.elsevier.com/solutions/engineering-village/content/compendex} 
& \multicolumn{3}{p{0.2\textwidth}|}{Keine 
booleschen\newline Ausdrücke möglich} \\
	\hline
	AIS Electronic Library (AISeL) \newline \url{http://aisel.aisnet.org/} & 
\multicolumn{3}{p{0.21\textwidth}|}{Keine 
booleschen\newline Ausdrücke möglich} \\
	\hline
	Zeitschriftendatenbank (ZDB) \newline 
\url{http://dispatch.opac.ddb.de/DB=1.1/srt=YOP/} & ? & ? & ? \\
	\hline
	IEEE Xplore \newline 
	\url{http://ieeexplore.ieee.org/Xplore/dynhome.jsp?tag=1} & 
	? & ? & ? \\
	\hline
	Springer-Online: Bücher/Beiträge des Springer Verlags \newline
	\url{http://www.springerlink.com} & $\surd$ &  &  \\
	\hline
Rechercheangebot der ULB 
\newline \url{http://www.ulb.tu-darmstadt.de/recherche/} & $\surd$ & 
$\surd$ & $\surd$ \\
\hline
%	WiSo Net: deutschsprachige Literatur zu Wirtschafts- und 
%	Sozialwissenschaften \newline \url{www.wiso-net.de} & ? & ? & ? \\
%	\hline
	%EBSCO: internationale wirtschafts-wiss. Zeitschriften \newline 
	%\url{http://search.ebscohost.com} & ? & ? & ? \\
	%\hline
\end{tabular}

\caption{Literaturdatenbanken und für welche Fragen sie herangezogen wurden. 
Quellen: \cite{exploring_the_factors} und \cite{formatvorlage}}
\label{tab:literaturdatenbanken}
\end{table}
% $\surd$

\begin{comment}
\subsubsection{Sonstiges}
\begin{itemize}
\item \textbf{Google Scholar:} Suchdienst für wissenschaftliche Recherchen 
(http://scholar.google.de)
\item \textbf{Verlagswebseiten} Recherche und den Zugriff auf Zeitschriften- 
und 
Zeitungsartikel und E-Books
\item \textbf{Webseiten von Unternehmen} für die Recherche von 
Unternehmensdaten 
und-statistiken sowie Unternehmensdatenbanken
\item \textbf{Webseiten von Bundes- und Landesbehörden sowie der EU}
 Statistisches Bundesamt (http://www.destatis.de)
\\Presse- und Informationsamt der Bundesregierung 
(http://www.bundesregierung.de)
\item \textbf{Webseiten von Marktforschungsinstituten}
(für Marktanteile und Verbraucheranalysen)
\item \textbf{Webseiten von Verbänden und Kammern}
Institut der deutschen Wirtschaft (http://www.deutsche-wirtschaft.de)
\end{itemize}
\end{comment}
